\documentclass[lettersize,journal]{IEEEtran}
\usepackage{amsmath,amsfonts}
% \usepackage{algorithmic}
\usepackage{array}
\usepackage[caption=false,font=normalsize,labelfont=sf,textfont=sf]{subfig}
\usepackage{textcomp}
\usepackage{stfloats}
\usepackage{url}
\usepackage{verbatim}
\usepackage{graphicx}
\usepackage{cite}

% to be able to draw some self-contained figs
\usepackage{tikz}

% inlined bib file
\usepackage{filecontents}

\usepackage[english]{babel}
\usepackage{blindtext}
% \usepackage[toc,page]{appendix}

\usepackage{xcolor}

\usepackage{color}
\usepackage{multirow}
\def\UrlBreaks{\do\/\do-}
% \usepackage[ruled,linesnumbered, vlined]{algorithm2e}
\usepackage{booktabs}
\let\labelindent\relax
\usepackage{enumitem} % for concise list env
\usepackage{xspace}
\usepackage{latexsym}
\usepackage{tabularx}
\usepackage{balance}
\usepackage{pifont}% http://ctan.org/pkg/pifont
\usepackage{soul}
\usepackage{siunitx}
\usepackage{amsthm}
\usepackage{tikz}
\usepackage{comment}
\usepackage{subfloat}
\usepackage{upgreek}
\usepackage{threeparttable}
\usepackage{makecell}
\usepackage{microtype}
\usepackage[shortcuts]{extdash}
\usepackage{footnote}
\usepackage{mdwlist}
\usepackage{adjustbox}
\usepackage{hyperref}
\hypersetup{
    colorlinks  =true,
    urlcolor     = black, %Colour for external hyperlinks
    linkcolor    = blue, %Colour of internal links
    citecolor   = blue %Colour of citations
}
\usepackage[noend]{algpseudocode}
\usepackage{algorithm}
\usepackage{algpseudocode}

\hyphenation{op-tical net-works semi-conduc-tor IEEE-Xplore}
% updated with editorial comments 8/9/2021

\newcolumntype{L}[1]{>{\raggedright\arraybackslash}p{#1}}
\newcolumntype{C}[1]{>{\centering\arraybackslash}p{#1}}
\newcolumntype{R}[1]{>{\raggedleft\arraybackslash}p{#1}}

%% self-defined commands
\newcommand{\paraf}[1]{\noindent\textbf{#1}}
\newcommand{\parabf}[1]{\medskip\noindent\textbf{#1}}
\newcommand{\sysname}{COSMOS\xspace}
% \algnewcommand{\LineComment}[1]{\Statex \(\) #1}
% \algnewcommand\algorithmicforeach{\textbf{foreach}}
% \algdef{S}[FOR]{ForEach}[1]{\algorithmicforeach\ #1\ \algorithmicdo}

\newcommand{\squishlist}{
   \begin{list}{$\bullet$}
    { \setlength{\itemsep}{0pt}      \setlength{\parsep}{3pt}
      \setlength{\topsep}{3pt}       \setlength{\partopsep}{0pt}
      \setlength{\leftmargin}{3.5mm} \setlength{\labelwidth}{1em}
      \setlength{\labelsep}{0.5em} } }
      
\newcounter{boxlblcounter}  
\newcommand{\squishnumlist}{
  \begin{list}{\arabic{boxlblcounter}.}
    { \usecounter{boxlblcounter} 
      \setlength{\itemsep}{0pt}      \setlength{\parsep}{3pt}
      \setlength{\topsep}{3pt}       \setlength{\partopsep}{0pt}
      \setlength{\leftmargin}{3.5mm} \setlength{\labelwidth}{1em}
      \setlength{\labelsep}{0.5em} } }

\newcommand{\squishend}{
    \end{list}  }
\newcommand*\circled[1]{\tikz[baseline=(char.base)]{
  \node[shape=circle,draw,inner sep=1pt] (char) {#1};}}

\newcommand\eg{{\em e.g.,}\xspace}
\newcommand\etc{{\em etc.}\xspace}
\newcommand\etal{{\em et al.}\xspace}

\newcommand{\py}[1]{{\color{red}#1}}
\newcommand{\xin}[1]{{\color{blue}(Xin: #1)}}
\newcommand{\junyi}[1]{{\color{cyan}[Junyi: #1]}}
\newcommand{\yunshan}[1]{{\color{orange}#1}}
\newcommand{\todo}[1]{{\color{red}[TODO: #1]}}
\newcommand{\xzl}[1]{\textbf{\color{violet}{(Xuanzhe: #1)}}}
\newcommand{\rui}[1]{\textbf{\color{purple}{(Rui: #1)}}}
\newcommand{\meng}[1]{{\color{blue}[Meng: #1]}}
\newcommand{\wm}[1]{{\color{black}#1}}
\newcommand{\yuxiang}[1]{{\color{blue}(Lin: #1)}}
\newcommand{\jys}[1]{{\color{blue}#1}}
\newcommand{\chao}[1]{{\color{orange}{Chao: #1}}}

\algdef{SE}[SUBALG]{Indent}{EndIndent}{}{\algorithmicend\ }%
\algtext*{Indent}
\algtext*{EndIndent}
\algnewcommand\algorithmicforeach{\textbf{for each}}
\algdef{S}[FOR]{ForEach}[1]{\algorithmicforeach\ #1\ \algorithmicdo}

\begin{document}

\title{\sysname: Taming Heterogeneity in Serverless Functions with \textbf{Co}mplementary \textbf{S}warming and \textbf{Mo}notonic \textbf{S}cale-up}

% \author{\pageref{lastpage} pages, \pageref{allpages} pages total}
\author{Yunshan Jia,
        Junyi Shu,~\IEEEmembership{Member,~IEEE},
        Chiheng Lou,
        Fangyue Liu,
        Chenbo Bi,
        and Xin Jin,~\IEEEmembership{Senior Member,~IEEE}
        % <-this % stops a space
        
\IEEEcompsocitemizethanks{
% \vspace{-5pt}
\IEEEcompsocthanksitem Yunshan Jia, Chiheng Lou, Fangyue Liu, Chenbo Bi, and Xin Jin are with the School of Computer Science, Peking University, Beijing, China.
\protect
E-mail: \{jiayunshan;\todo{};\todo{};\todo{};xinjinpku\}@pku.edu.cn.
\IEEEcompsocthanksitem Junyi Shu is with University of California, Los Angeles, California, USA.
\protect
E-mail: kyleshu@ucla.edu.

This work was supported in part by the National Natural Science Foundation of China under Grant 62172008 and 62325201, and in part by the National Natural Science Fund for the Excellent Young Scientists Fund Program (Overseas). (Corresponding author: Junyi Shu.)
}% <-this % stops an unwanted space
}

% The paper headers
\markboth{IEEE/ACM Transactions on Networking}%
{Shell \MakeLowercase{\textit{et al.}}: A Sample Article Using IEEEtran.cls for IEEE Journals}

% \IEEEpubid{0000--0000/00\$00.00~\copyright~2021 IEEE}
% Remember, if you use this you must call \IEEEpubidadjcol in the second
% column for its text to clear the IEEEpubid mark.

\maketitle

\begin{abstract}

Serverless computing is an increasingly popular paradigm for cloud applications, offering scalability and pay-per-use billing. However, serverless platforms must efficiently manage resources for a multitude of diverse functions and requests with different inputs. Conventional approaches often ignore the resource requirement differences between functions and individual requests, leading to resource fragmentation, poor cluster utilization, and high overhead from instance initializations.

This paper introduces COSMOS, a novel serverless computing system that improves efficiency by exploiting the complementary resource usage patterns across different functions. Our approach is founded on two key optimizations. First, we co-locate functions with complementary resource requirements to maximize node utilization, while simultaneously grouping instances of the same function to improve locality and reduce associated overhead. Second, we leverage the emerging capability to resize an instance's resource allocation, allowing our system to reuse existing warm instances for requests with heterogeneous resource demands and minimize instance initializations. Experiments on real-world workloads show our system improves resource utilization by up to 55.53\% (CPU) and 67.01\% (memory), and increases overall cluster throughput by up to 2.03$\times$ compared to state-of-the-art serverless computing systems.

\end{abstract}

\begin{IEEEkeywords}
Serverless, scheduling, placement and routing.
\end{IEEEkeywords}

\section{Introduction}
\label{sec:introduction}

Serverless computing is an emerging cloud paradigm that abstracts away server management, allowing developers to focus solely on application logic~\cite{CACM19:Rise, TR19:Berkeley, CACM21:Serverless}. In this model, developers deploy programs as cloud functions, and the platform automatically manages the underlying infrastructure, including resource provisioning and auto-scaling. The paradigm's simplicity and efficiency have led to its widespread adoption in domains ranging from web applications and data processing pipelines~\cite{Doc:AzureWebApp, NSDI19:Shuffling, OSDI18:Pocket} to machine learning workloads~\cite{SC20:Batch, OSDI21:Dorylus}. For cloud providers, however, delivering on the promise of serverless requires extreme resource efficiency across a multi-tenant fleet of servers.

The prevailing approaches to efficient resource management typically pursue simple objectives like balancing load~\cite{SoCC22:Hermod, Doc:Knative, Doc:OpenFaaS} or maximizing the reuse of warm instances~\cite{ASPLOS21:FaasCache, ASPLOS24:RainbowCake, ASPLOS22:IceBreaker, ATC20:SitW, SoCC22:Hermod, Doc:OpenWhisk}. However, these approaches are resource-inefficient because they ignore the diverse resource footprints of different functions. A scheduler might place several memory-intensive functions on the same node, rapidly exhausting its memory while leaving CPU cores idle. This creates stranded resources, leading to poor cluster-wide utilization because the node cannot accept new functions that require memory, even though it has ample CPU capacity.

Furthermore, existing resource allocation strategies operate at the function level, where each function instance is allocated a static resource capacity~\cite{Doc:AWS_Lambda_Mem, Doc:Azure_Functions_Hosting, Doc:GCP_CloudRun_Mem, ASPLOS21:FaasCache, JinZXZHLJ23, zhang2024jolteon, ZhangTKCS21, ORION_MahgoubYSECB22,AQUATOPE_ZhouZD23}. This strategy is predicated on the assumption that a function exhibits uniform resource demands across all its invocations. However, this one-size-fits-all approach often leads to significant resource inefficiency, as it fails to adapt to dynamic, input-dependent workload variations.

We confirm these inefficiencies by conducting a comprehensive analysis over a wide range of real-world serverless applications (\S\ref{sec:background:demand-analysis}). Our analysis shows that i) the CPU-to-memory ratios (core/GB) of different functions range from 0.09 to 9.20, with a CV (Coefficient of Variation) of 62.43\% which leads to sigficant resource stranding with existing placement strategies; ii) and 41\% of functions have resource usage with CV exceeding 50\% which forces over-provisioning of resources under static resource allocation. 

To overcome these inefficiencies, we present \sysname, a serverless computing system that handles heterogeneity of functions and requests with a two-pronged solution. At the inter-server level, we introduce complementary swarming, a function placement strategy that mitigates resource stranding by co-locating functions with opposing resource requirements while preserving locality. At the intra-server level, we design a monotonic scale-up request scheduling policy that reorders the request queue for a function to enforce monotonic resource adjustment which avoids the high overhead of instance restarts. Together, these techniques allow \sysname to significantly improve cluster utilization.

The complementary swarming placement strategy operates by first classifying functions based on their resource profiles (i.e., CPU usage and memory usage). It then employs a packing algorithm that treats CPU and memory as distinct dimensions, pairing functions with opposing resource needs onto the same physical node. This co-location of complementary workloads ensures that a node's full capacity is utilized, preventing one resource dimension from becoming a bottleneck while another sits idle and thereby maximizing server density.

The ability to manage resources at the request level is enabled by recent techniques that permit the dynamic, in-place adjustment of a container's resource capacity\cite{Doc:K8s_Pod_Resize}. The monotonic scale-up scheduling algorithm is designed to leverage this capability. \sysname organizes incoming requests into multiple queues, each sorted by monotonically increasing resource demands. A warm instance processes requests sequentially from a single queue, ensuring its resource capacity is almost always scaled up. Costly cold start operations are only performed infrequently, thereby amortizing their high cost across many requests.

The key contributions of this paper are as follows:
\begin{itemize}
    \item We conduct a comprehensive characterization of real-world serverless workloads, identifying and quantifying the resource inefficiency caused by function and request heterogeneity.
    \item We design \sysname, a serverless computing system that co-locates functions with complementary resource requirements and dynamically adjust resources at the request level with complementary swarming for placement and monotonic scale-up for scheduling.
    \item We implement and evaluate \sysname on real-world workloads, demonstrating up to 55.53\% higher CPU utilization, 67.01\% higher memory utilization, and 103.71\% greater cluster throughput compared to state-of-the-art serverless systems.
\end{itemize}

% \todo{maybe need a table to compare \sysname with existing solutions}

%Resource management in serverless computing has been an extensively studied problem~\cite{FaaSCache_ASPLOS21}, with numerous solutions proposed in the literature\todo{citation}.
%Most existing approaches operate at the function level; that is, resources are allocated to function instances, which further select requests to process.
%A key assumption underlying these approaches is that a serverless function exhibits consistent resource demands across all invocations.
%
%However, recent analyses of real-world serverless workloads have revealed significant skew in resource consumption across different requests for the same function\todo{citation}.
%Our analysis (\S\ref{sec:demand-analysis}) also confirms that the resource usage of a serverless function can vary by up to XXX times\todo{variation} across different requests.
%This high variability leads to significant inefficiencies for conventional, function-level policies.
%To guarantee reliable performance, these policies must provision resources for a function's peak (i.e., worst-case) demand, resulting in substantial resource wastage for the vast majority of less-demanding requests. 
%Additionally, we find that this input-dependent resource usage is not only skewed but also highly predictable, achieving an accuracy of XXX\%\todo{prediction accuracy}.
%
%Concurrently, techniques that allow for dynamic adjustment of a container's resource capacity have been proposed\todo{citation(k8s, escra, etc.)}.
%These techniques make it possible to alter the resource allocation of a running container in-place, thereby avoiding the costly overhead of resource reallocation.
%
%Capitalizing on these developments, we propose a \textit{request-level} resource allocation policy that provisions the exact amount of resources for each request.
%The key idea is to predict the resource demand of each incoming request and dynamically adjust the target container's resource capacity accordingly.
%Such a fine-grained strategy ensures that requests are processed using only the necessary resources, which can significantly reduce cluster-wide resource wastage.
%
%However, realizing an effective request-level resource management system introduces two challenges.
%First, to achieve high resource utilization through dynamic resource reallocation, the system must rely on oversubscription,
%where the sum of maximum possible resources required for colocated containers exceeds the physical capacity of the server.
%Given the high variability in per-request resource demands, a high oversubscription ratio leads to resource contention and frequent container evictions.
%This forces the system to adopt a low oversubscription ratio, negating much of the potential efficiency gains from request-level resource allocation.
%
%Second, dynamic resource adjustment mechanisms introduce non-negligible overhead.
%Specifically, while increasing a container's resource capacity is a relatively fast operation, decreasing it is often much slower.
%For example, reducing the memory capacity of a container requires reclaiming memory pages, which incurs substantial latency.
%As resource adjustment occurs on the critical path of request execution, this latency directly increases the end-to-end request handling time.
%
%To address these challenges, we propose \sysname, a serverless resource manager that allocates resources at the request level.
%To mitigate the risk of eviction from oversubscription, \sysname adopts a XXX\todo{placement algorithm name} placement strategy.
%This strategy classifies requests into distinct tiers based on their predicted resource demands and specializes containers to handle requests from specific tiers.
%Moreover, this strategy incorporates a XXX\todo{Other feature of placement algorithm. E.g., coloring.} technique to maximize server utilization by colocating containers with complementary resource requirements.
%
%
%\sysname further tackles the overhead of resource adjustment with a XXX\todo{scheduling algorithm name} request scheduling algorithm.
%This algorithm organizes incoming requests into monotonic queues, where requests are sorted by increasing resource demand.
%By processing the requests in a queue sequentially, a container's resource capacity only needs to be incrementally increased.
%A costly resource-decreasing operation is only required when the system finishes one queue and switches to another with a lower starting resource requirement.
%This design reduces the number of resource-decreasing operations, thereby significantly mitigating its impact on overall latency.
%
%Experiments show that \todo{summary of experiments results.}
%
%\todo{summary of contributions.}
\section{Background and Motivation}
\label{sec:background}

\subsection{Resource Management of Serverless Computing}
\label{sec:background:resource-management}

% 1. serverless的资源管理模式: 静态资源分配管理
% 把 resource allocation 分为两个步骤,一个是节点-level的资源分配,即运行时的核数和内存,另一个cluster-level的资源分配,即是函数容器副本扩缩容和放置,即决定副本数量、并将容器放置到具体的节点。

Serverless computing has emerged as a promising cloud paradigm, allowing developers to focus on application logic by abstracting away infrastructure tasks such as provisioning, scaling, and maintenance. In this event-driven model, users simply provide code (functions) with basic resource specifications and triggers. The platform then orchestrates the entire lifecycle\textemdash provisioning sandboxed environments, performing on-demand autoscaling, and managing pay-per-use billing. Consequently, efficient resource management within these platforms is essential to ensure high performance, maximize utilization, and minimize costs.

In practice, resource management in serverless platforms encompasses two primary dimensions: i) per-instance resource configuration, which defines the resource limits for individual function sandboxes; and ii) per-function orchestration, which governs autoscaling for instance count and placement for node~selection.

\yunshan{\parabf{Per-instance resource allocation.}
% 1) 静态的运行时资源配置:介绍 lambda 支持指定固定资源规格,及其对延迟和开销的影响,所以选取一个所有请求都不违背qos的配置
Existing serverless computing systems from both industry~\cite{Doc:Azure_Functions_Hosting,
Doc:GCP_CloudRun_Mem, aws_lambda} and academia~\cite{FaaSCache, shahrad2020serverless, wang2018peeking, pu2019shuffling}
allocate a uniform amount of resources for all instances of a given function. In AWS Lambda, a developer specifies a memory size (e.g., 128 MB--10,240 MB), from which the platform derives proportional vCPU and network bandwidth shares~\cite{aws_lambda}. Developers may also set ephemeral storage per instance (default 512 MB, configurable up to a few gigabytes). These limits remain fixed throughout the sandbox lifetime. Similar static per-instance specifications are used by Azure Functions~\cite{Doc:Azure_Functions_Hosting} and Google Cloud Run~\cite{Doc:GCP_CloudRun_Mem}. This approach presumes uniform resource demand across function invocations. Consequently, configuration optimizations focus on profile-guided parameter tuning~\cite{aws_lambda_power_tuning,FaaSCache, shahrad2020serverless, zhang2024jolteon, JinZXZHLJ23}: by profiling across discrete configurations, they derive latency–cost curves to select the cheapest configuration that satisfies a given SLO, thereby minimizing cost under QoS constraints.


\parabf{Instance scaling and placement.}
% 2)根据静态资源配置进行容器扩缩容放置:现有的 serverless 系统大多基于静态资源配置进行容器放置调度
% 仅容器启动、离开时比较资源:只需维护节点已分配资源和未分配资源总计,任务到来时可以分配到未分配资源>静态请求资源的节点上,更新已分配资源、未分配资源,函数容器销毁时释放资源、更新已分配资源、未分配资源即可
Existing serverless systems perform scaling~\cite{aws_scaling_window, aws_scaling_eq, knative_scaling} and placement based on a static per-function resource requirement. While autoscalers adjust replica counts in response to load, schedulers employ a simple resource accounting model where each node tracks allocated versus available capacity. The resource counters are updated only during lifecycle events\textemdash decremented upon placement and incremented upon termination\textemdash ignoring actual usage during execution. A container is admitted to a node only if the available capacity meets the function’s static requirement (e.g., memory, vCPU). For instance scaling and placement, recent research has pursued two main optimizations: (i) for autoscaling, using prediction to right-size replica counts and prewarm functions~\cite{shahrad2020serverless, ORION_MahgoubYSECB22}; and (ii) for placement, improving efficiency through cache-aware~\cite{SOCK_OakesYZHHAA18}, heterogeneity-aware~\cite{RoyPT22}, and data-affinity strategies~\cite{Wukong_CarverZWAWC20, ORION_MahgoubYSECB22}.

Taken together, the prevailing serverless paradigm relies on modeling each function with a static, uniform resource profile. However, our characterization reveals that this static paradigm conflicts with a dual-layered heterogeneity: diverse resource profiles across functions and variable demands within each function. In the next two subsections, we first quantify these heterogeneities (\S\ref{sec:background:demand-analysis}) and then analyze the system-level challenges they pose to dynamic resource allocation (\S\ref{sec:background:challenges}).}

% 2. serverless 函数资源需求分析
\subsection{Analysis of Resource Requirements of Functions}
\label{sec:background:demand-analysis}

\begin{figure}[t]
  \centering
  \includegraphics[width=0.95\linewidth]{figures/background/cpu_memory_ratio_histogram.pdf}
  \caption{Heterogeneous CPU:memory ratios across functions.}
  \label{fig:cpu_mem_ratio_hist}
\end{figure}

% 1) 函数级别需求分布
\paraf{Resource Usage Variation Across Functions.}
The resource demands of serverless functions are highly heterogeneous. We quantify this heterogeneity by analyzing the execution traces of 162 real-world functions from the open-source Huawei dataset~\cite{huawei_dataset}, calculating the average CPU-to-memory ratio (core/GB) for each. Figure~\ref{fig:cpu_mem_ratio_hist} presents the distribution of these ratios across all functions.\footnote{The statistical results shown are from [Request Table Module, Region 4] part in dataset~\cite{huawei_dataset}. We observed similar distributions across different parts of the dataset.} The distribution is highly dispersed, with ratios ranging from 0.09 to 9.20 and a high CV of 62.43\%, indicating that some functions are intensely compute-intensive while others are predominantly memory-intensive. This substantial variation in intrinsic resource profiles implies that a scheduler unaware of such differences is prone to creating resource fragmentation, where one resource type is exhausted on a node while another remains underutilized.

\begin{figure}[t]
    \centering
    \includegraphics[width=0.95\linewidth]{figures/background/cdf_cpu_cv_regions.pdf}
    \caption{Resource usage variations of serverless function.}
    \vspace{-15pt}
\label{fig:usage_variation}
\end{figure}

% 2) 调用级别资源占用
\paraf{Resource Usage Variation Across Invocations.}
Resource demand also varies significantly across different invocations of the same function. We compute the coefficient of variation (CV) for the CPU usage of each function across its invocations. The distribution of these CV values is presented in Figure~\ref{fig:usage_variation}. Our analysis shows that variability is the norm rather than the exception: for 54\% of functions, the CV exceeds 50\%, and an additional 26\% have a CV between 15\% and 50\%. This widespread and often substantial variability directly contradicts the assumption of uniform demand underlying static allocation. Consequently, any static resource allocation must provision for the peak demand of such variable functions, inevitably leading to low average utilization.

% % 2) 现有调度方案的问题
% \parabf{Drawback of existing solutions: Static resource allocation waste resources.}
% Static, uniform per‑instance sizing ignores intra‑function resource usage variability.
% When set to a high tier, most invocations are overprovisioned, wasting CPU/memory
% and inflating cost. When set to a low tier, peak‑demand invocations are
% underprovisioned, causing throttling, queuing, memory overflow.
% Fixed sizing also worsens data locality and resource fragmentation.

% 3) Opportunity: 函数动态运行时资源调整 
% 现有容器框架支持动态运行时资源分配(具体机制或者接口),以及其好处(节省资源减少浪费)
\parabf{Opportunity: Dynamic in‑place runtime resource adjustment}
A key opportunity to address the intra-function variability lies in recent container 
runtime enhancements that support dynamic, in-place resource adjustment. Specifically, 
platforms like Kubernetes now allow the resource allocations (CPU and memory) of 
a running pod to be updated without restarting it, via mechanisms such as the Pod 
resize subresource and live cgroup updates~\cite{Doc:K8s_Pod_Resize, Doc:K8s_CRI_Update}.
This capability enables a container's assigned resources to be aligned in real 
time with the actual demand of the request it is processing. It thus provides a 
foundational mechanism to move beyond static per-function allocation, toward a 
fine-grained model where resources can be expanded or contracted per invocation. 
It absorbs demand bursts, reduces stranded resources from over-provisioning, 
and lowers the risk of SLO violation\textemdash while preserving performance 
benefits of a warm sandbox.

% Kubernetes enables in‑place vertical scaling without pod recreation via the Pod 
% resize subresource, allowing live updates to both CPU and memory resource requests 
% and limits. CPU resources can be adjusted by altering cgroup CPU shares and quotas, 
% allowing the kubelet to allocate CPU cycles according to usage metrics and policy 
% constraints. Similarly, memory scaling is achieved by modifying cgroup limits,
% which dynamically changes the memory allocation for a running container without 
% stopping it.\cite{Doc:K8s_Pod_Resize, Doc:K8s_CRI_Update}. Dynamically aligning 
% each invocation's resource demands with its sandbox allocation absorbs bursts and 
% handles intra-function heterogeneity, improving utilization and reducing 
% overprovisioning and SLO risks. This ensures that all resources can be increased 
% or decreased to respond to real-time demands. A serverless control plane can 
% utilize these features to dynamically adjust sandboxes for both CPU and memory, 
% optimizing resources while preserving state and improving cost and latency efficiency.

% 3. 动态资源分配的挑战

\subsection{Challenges in Dynamic Resource Allocation for Functions}
\label{sec:background:challenges}

% 1)挑战1:副本放置挑战:动态资源分配打破了原有的静态资源分配假设,容器的资源需求不再是固定的,不能仅在容器启动和销毁时考虑节点资源分配
% strawman solution:发生扩容剩余节点资源不足时直接evict其他同节点容器,但带来缓存损失,且可能放大(相对较大规格的容器扩容,可能一下需要evict多个小规格容器)
% appeal for Minimal-Disruption Placement Scaling with multi-resource balance awareness

\paraf{Challenge 1: Placement under dynamic in-place resizing.} Dynamic in-place resizing disrupts the fixed-size bin-packing model: a sandbox’s footprint may expand while active, making node feasibility assessments at start/stop times inadequate. A simplistic approach is to evict co-located sandboxes to make space when upscaling faces insufficient capacity. However, this naive strategy risks evicting other co-located function instances with increased resource demand. This issue is exacerbated when a large upsizing occurs under constrained resource availability, resulting in eviction amplification: a single significant resize event can force the eviction of multiple smaller co-located sandboxes, thereby hurting cache locality and increasing tail latency. Additionally, fragmentation can occur when resource demand descreases. These problems may lead to low resource utilization and application throughput.

\yunshan{Therefore, an intelligent placement strategy must anticipate and mitigate the disruptive effects of resizing. Its goal is twofold: i) to balance CPU and memory usage per node from the outset, preventing resource fragmentation and minimizing the need for disruptive migrations; and ii) when migrations are unavoidable, to execute them in a manner that minimizes overall sandbox disruption. By promoting managed, minimal churn, such a strategy safeguards warm states (maximizing reuse) and maintains high packing efficiency and throughput even during dynamic resizing.}
% todo: 请按需加图

% \begin{figure}[t]
%   \centering
%   \begin{minipage}[t]{0.48\linewidth}
%     \centering
%     \includegraphics[width=\linewidth]
%     {figures/background/challenge2_eviction_amplification.png}
%     \vspace{2pt}
%     \small Left: Upsizing a large sandbox may evict many small ones
%     (amplification).
%   \end{minipage}\hfill
%   \begin{minipage}[t]{0.48\linewidth}
%     \centering
%     \includegraphics[width=\linewidth]
%     {figures/background/challenge2_stranding.png}
%     \vspace{2pt}
%     \small Right: Ignoring multi‑resource balance strands CPU or memory.
%   \end{minipage}
%   \caption{Challenge 2: Dynamic in‑place scaling complicates placement under
%   tight capacity.}
%   \label{fig:challenge2_placement}
% \end{figure}


% 3)挑战3:资源缩容本身有成本挑战:扩容成本低,但缩容成本高。对于请求量没那么大的函数有难度。
%Second, dynamic resource adjustment mechanisms introduce non-negligible overhead.
%Specifically, while increasing a container's resource capacity is a relatively fast operation, decreasing it is often much slower.
%For example, reducing the memory capacity of a container requires reclaiming memory pages, which incurs substantial latency.
%As resource adjustment occurs on the critical path of request execution, this latency directly increases the end-to-end request handling time.
% add overhead detail

% 对于请求量没那么大的函数
% strawman 1: 起很多副本,细致应对不同资源需求,但导致大量浪费
% strawman 2:只起少量副本,一会儿扩一会儿缩,但频繁调整带来高开销
% appeal for(opportunity): Vertical Scaling-Based Task Scheduling Algorithm,即按请求资源量从小到大reorder请求,使得单容器能从小到大单调递增扩容,然后只减少一次到很小,循环往复
\paraf{Challenge 2: High overhead of downsizing.} Dynamic in-place scaling is asymmetric: upsizing is swift, but downsizing incurs high costs. Increasing the memory allocation takes effect immediately, whereas reducing it triggers cgroup reclamation, page-cache eviction, and compaction, potentially stalling both runtime and kernel. \yunshan{Since resizing affects the request path, this slow reclaim process increases tail latency and queueing, which in turn lowers overall node throughput even when average capacity suffices.}

Two baseline strategies struggle to handle bursty, diverse functions effectively. The first approach is to over-provision numerous replicas, each fixed to a narrow size band, to avoid the need for resizing. However, this leads to significant resource fragmentation and poor utilization due to excessive headroom. The second approach is to maintain fewer replicas and resize them dynamically for every request. Unfortunately, such frequent adjustments incur high memory reclamation overheads, disrupt page caches and JIT states, and increase scheduler interference. Ultimately, both strategies degrade system throughput, negating the theoretical advantages of dynamic scaling.

\yunshan{Therefore, a practical dynamic scaling scheduler must achieve the following goal: it should adapt resource allocation to variable request-level demand without incurring frequent downsizing overheads. This necessitates a mechanism that can batch or reorder requests to create resource usage patterns that are conducive to monotonic (upward-only) adjustments within short time windows, thereby amortizing reclamation costs and preserving warm sandbox state.}

\begin{figure}[t]
  \centering
  \subfloat[]{\includegraphics[trim=0 80pt 400pt 0, clip, width=0.47\linewidth]{figures/background/vertical.pdf}}
  \hfil
  \subfloat[]{\includegraphics[trim=400pt 80pt 0 0, clip, width=0.47\linewidth]{figures/background/vertical.pdf}}
  \caption{In-place scaling-based task scheduling mitigates downsizing overhead
  by reordering requests. (a) Naïve resizing, Cost = 12. (b) Monotonic Scale-up, Cost = 8. \todo{chaneg to over-provision \& resizing}}
  \vspace{-15pt}
  \label{fig:vertical_scaling_scheduling}
\end{figure}

% # 本文优化目标(场景:高负载、高并发,核心目标:资源利用率和任务吞吐量)
% \section{Algorithm Design}

\section{\sysname Overview}
\label{sec:overview}

{
\begin{figure}[t]
  \centering
  \includegraphics[width=\linewidth]{figures/overview.pdf}
  \vspace{-15pt}
  \caption{\sysname overview. \junyi{put multiple users in the figure}}
  \vspace{-15pt}
  \label{fig:overview}
\end{figure}
}

Figure~\ref{fig:overview} illustrates the overall architecture of \sysname. The scheduling framework operates at two levels: inter-node request dispatching and intra-node request scheduling. \junyi{should probably unify task/request, server/node to avoid confusing people} At the cluster level, incoming user requests are first directed to a Request Dispatcher, which monitors traffic patterns. These traffic statistics, along with function metadata and performance metrics, are forwarded to the Instance Placement Manager. The manager periodically executes an algorithm to generate a placement plan, which is returned to the Request Dispatcher to guide request-to-node assignments.
Once assigned to a worker node, requests are entered into queues managed by the Worker Scheduler. This component maintains separate queues and priorities for different functions, while orchestrating instances within the node to handle request execution and in-place resource scaling. To enable this, \sysname introduces two key techniques: Complementary Swarming Instance Placement and Monotonic Scale-Up Task Scheduling.

\parabf{Complementary Swarming Instance Placement.} \junyi{based on the description, it sounds more like request placement instead of instance placement}
\sysname controls instance placement by establishing a mapping between a swarm of requests from the same function sharing similar resource demands range to worker nodes. This mapping dictates how requests are distributed across the cluster. Using a heuristic algorithm, \sysname ensures that the collective resource demand of the function mix on each worker node aligns with the node's capacity, thereby maximizing utilization. Simultaneously, the system minimizes the number of distinct functions per node to increase instance reuse. For high-demand functions requiring deployment across multiple servers, \sysname employs request-level load splitting to distribute the workload. By controlling the optimization step size, \sysname guarantees algorithm convergence and keeps computation time under 100 milliseconds.

\parabf{Monotonic Scale-up Request Scheduling.}
\sysname designs its worker-level scheduling algorithm around the restart-free in-place pod scaling feature of Kubernetes. The scheduler maintains dual request queues, both sorted in ascending order of resource demand. This ordering ensures that, within a time window, the resource requirements of tasks processed by an instance increase monotonically. Consequently, a pod can be scaled up incrementally to meet the growing demands of successive requests without over-provisioning, thereby increasing resource utilization without incurring restarts. To prevent starvation, \sysname determines queue priorities by holistically considering task waiting time, cold-start overhead, and resource requirements. By strategically reordering tasks to extend queue lengths under priority constraints, \sysname effectively boosts resource utilization and reduces the frequency of cold starts.
\section{\sysname Design}

\sysname employs a two-level hierarchical approach for scheduling function instances 
and requests. At the inter-server level, complementary swarming placement groups 
and places functions with opposing CPU/memory needs onto the same server, maximizing 
node utilization and reuse. At the intra-server level, monotonic scale-up scheduling 
reorders each function’s requests by increasing resource demand, allowing a single 
sandbox to scale up incrementally and avoid restarts.

\subsection{Complementary Swarming Instance Placement}
\label{sec:design:placement}

The complementary swarming placement algorithm addresses the scheduling challenge 
posed by heterogeneous function resource demands, as quantified in 
\S\ref{sec:background:demand-analysis}. It pursues a dual objective: i) maximizing 
cluster-wide resource utilization by co-locating functions with complementary CPU 
and memory profiles on the same server, thereby mitigating resource stranding; 
and ii) enhancing instance reuse by deliberately narrowing the set of distinct 
functions served per node. To achieve this, the algorithm proceeds iteratively. 
In each round, it assigns the workload of a function to one or more worker nodes 
based on the function‘s resource profile and, when necessary, rolls back partial 
assignments of other already-placed functions to improve complementary packing. 
For functions requiring distribution across multiple nodes, the algorithm performs 
fine-grained load splitting according to the function resource demand distribution. 
The algorithm outcome is a stable task-to-node mapping that balances complementary 
packing with equitable load distribution.

\subsubsection{Modeling Node Capacity and Function Demand}
\label{subsec:modeling}

A precise formulation of the placement problem requires clear representations of 
both node resource capacity and function resource demand.

\textbf{Node Resource Capacity.} We model a worker node $N_j$’s available capacity 
as a two-dimensional resource vector $\mathbf{C}_j = \left(C_j^{\text{cpu}}, C_j^{\text{mem}}\right)$, 
representing its spare CPU cores and memory (in GB). Similarly, we denote by 
$\mathbf{C}_j^{\text{idle}} = \left(C_j^{idle,cpu}, C_j^{idle,mem}\right)$ the 
vector of idle (currently unallocated) resources on $N_j$ at any given moment. 
For simplicity and clarity of exposition, we focus on CPU and memory as the 
primary bottleneck resources. Our algorithm can be extended to incorporate other 
resources (\emph{e.g.}, network bandwidth) by increasing the dimensionality of 
the capacity vector. To quantify the efficiency of a placement, we define the 
\textbf{overall resource utilization} of a node \(N_j\) as the sum of its normalized 
CPU and memory utilization:
\[
U(N_j) = \frac{U_j^{\text{cpu}}}{C_{\text{total}}^{\text{cpu}}} + \frac{U_j^{\text{mem}}}{C_{\text{total}}^{\text{mem}}}
\]
where \(U_j^{\text{cpu}}\) and \(U_j^{\text{mem}}\) are the allocated resources 
on \(N_j\), and \(C_{\text{total}}^{\text{cpu}}\) and \(C_{\text{total}}^{\text{mem}}\) 
are the \textbf{total capacities of the entire cluster}. This definition inherently 
balances both resource dimensions. A placement or swapping operation is considered 
beneficial if it increases \(U(N_j)\). (The formulation can be extended with 
resource-specific weights if needed.)

\textbf{Function Resource Demand.} Modeling the resource demand of a function is 
more nuanced. We characterize it along three orthogonal dimensions:

\begin{enumerate}
    \item \textbf{Demand Ratio ($r_i$)}: The intrinsic CPU-to-memory ratio required 
    by a single execution of function $F_i$. The value of $r_i$ quantitatively 
    indicates a function’s resource affinity: a relatively small $r_i$ signifies 
    memory-intensive behavior, whereas a relatively large $r_i$ corresponds to 
    compute-intensive behavior. To reduce problem complexity, we assume this 
    ratio remains invariant across invocations of the same function.

    \item \textbf{Demand Volume ($\mathbf{V}_i$)}: The aggregate resource demand 
    vector for function $F_i$ within the scheduling window, denoted as 
    $\mathbf{V}_i = \left(V_i^{\text{cpu}}, V_i^{\text{mem}}\right)$. Here, $V_i^{\text{cpu}}$ and 
    $V_i^{\text{mem}}$ represent the predicted total CPU (\emph{e.g.}, in core-seconds) 
    and memory (\emph{e.g.}, in GB-seconds) required to process all expected 
    invocations of $F_i$ in that window, respectively. While the accurate 
    prediction of $V_i$ is orthogonal to our placement algorithm (and its accuracy 
    naturally affects overall performance), our design accepts such a predicted 
    vector as input.

    \item \textbf{Demand Distribution ($D_i$)}: The statistical distribution of 
    per-invocation resource requirements, derived from historical traces. As shown 
    in Fig.~\ref{fig:usage_patterns}, these distributions exhibit diverse shapes 
    (\emph{e.g.}, quasi-normal, long-tailed). $D_i$ is crucial for fine-grained 
    load splitting when a function must be spread across multiple nodes.
\end{enumerate}
Together, the tuple $\left<r_i, \mathbf{V}_i, D_i\right>$ forms a comprehensive 
demand profile for each function $F_i$, enabling our algorithm to reason about 
complementary placement and proportional workload distribution simultaneously.

{
% \setlength{\belowdisplayskip}{-140pt}
\begin{algorithm}[t]
    \caption{\mbox{Complementary swarming placement.}}
    \noindent \textbf{Data:}
    \begin{algorithmic}[1]
    \Statex \begin{itemize}
        \setlength{\itemindent}{.0em} % 控制缩进
        \item Pending assignment vector \(\mathbf{R} \in [0,1]^n\).
        \item Assignment matrix \(\mathbf{L} \in [0,1]^{m \times n}\).
    \end{itemize}
    \Require $ $
    \Statex \begin{itemize}
        \setlength{\itemindent}{.0em} % 控制缩进
        \item Node set \(\mathcal{N} = \{N_1, \dots, N_m\}\).
        \item Function set \(\mathcal{F} = \{F_1, \dots, F_n\}\).
    \end{itemize}
    \Ensure Optimized placement $\mathcal{P}: \mathcal{F} \times [0,1] \to \mathcal{N}$.
    
    \State \textbf{// Step 1: Initialize state \& normalize function demands}
    \State \(\mathbf{L} \gets \mathbf{0}_{m \times n}\), \(\mathbf{R} \gets \mathbf{1}_n\)
    \For{$F_i \in \mathcal{F}$}
        \State $\mathbf{V}_i \gets \mathbf{V}_i \times \frac{\sum_{j=1}^{m} \mathbf{C}_j}{\sum_{i=1}^{n} \mathbf{V}_i}$ \Comment{Scale demand proportionally}
    \EndFor
    \State
    
    \State \textbf{// Step 2: Main optimization loop}
    \While{$\exists \; F_i \text{ s.t. } R[i] > 0$} \Comment{Placement incompletement}
        \State $F_a \gets \text{RandomChoice}(\{F_i \mid R[i] > 0\})$ 
        \For{$N_j \in \mathcal{N}$}
            \State \textbf{// Assignment via free capacity}
            \State $\alpha_j^{\text{free}} \gets min(\frac{C_j^{\text{idle,cpu}}}{V_i^{\text{cpu}}}, \frac{C_j^{\text{idle,mem}}}{V_i^{\text{mem}}})$
            \State \textbf{// Assignment via beneficial swaps}
            \State $(\alpha_j^{\text{swap}},\; \mathcal{E}_j) \gets \text{SwapCapacity}(F_a, N_j, \mathbf{L}, \mathbf{C}_j^{idle})$
            \Comment{$\mathcal{E}_j$: list of $(F_k, \beta_k)$ evicted}
            \State $\alpha_j^{\text{max}} \gets min(\alpha_j^{\text{free}} + \alpha_j^{\text{swap}}, R[a])$
            \State $\Delta U_j \gets \text{UtilGain}(F_a, \alpha_j^{\text{swap}}, \mathcal{E}_j, \mathbf{C}_j, \mathbf{C}_{\text{total}})$
        \EndFor

        \State \textbf{// Decision \& Assignment (ignoring edge cases)}
        \For{$N_j \in \mathcal{N}$, \textbf{descending} order of $\Delta U$}
            \If{$R[a] \leq 0$} \textbf{break} \EndIf
            \State $L[j, a] \gets L[j, a] + \alpha_j^{\text{max}}$, $R[a] \gets R[a] - \alpha_j^{\text{max}}$
            \For{$(F_k, \beta_k)$ in $\mathcal{E}_j$}
                \State $L[j, k] \gets L[j, k] - \beta_k$, $R[k] \gets R[k] + \beta_k$
            \EndFor
        \EndFor
    \EndWhile
    \State

    \State \textbf{// Step 3: Generate $\mathcal{P}$ from $L$}
    \For{$F_i \in \mathcal{F}$}
        \State $p_{\text{start}} \gets 0.0$ \Comment To store workload percentile range
        \For{$N_j \in \mathcal{N}$, where $L[j,i] > 0$, in arbitrary order}
            \State // We use half‑open intervals $[p_s, p_e)$, except the last interval which is closed at $1$.
            \State $\mathcal{P}(F_i, [p_{\text{start}}, p_{\text{start}} + L[j,i])) \gets N_j$ 
            \State $p_{\text{start}} \gets p_{\text{start}} + L[j,i]$
        \EndFor
    \EndFor
    \State \Return $\mathcal{P}$


\end{algorithmic}
\label{alg:placement}
\end{algorithm}
}

\subsubsection{Complementary Swarming Placement Algorithm}
\label{subsec:placement-algorithm}

Algorithm~\ref{alg:placement} presents the complementary swarming placement 
algorithm. It takes as input the set of nodes $\mathcal{N}$ with their capacity 
vectors $\mathbf{C}_j$, and the set of functions $\mathcal{F}$ with their demand 
profiles $\langle r_i, \mathbf{V}_i, D_i \rangle$. The algorithm outputs a placement 
mapping $\mathcal{P}: \mathcal{F} \times [0,1] \to \mathcal{N}$. This mapping 
determines task dispatch: when a function's workload is split across multiple 
nodes, $\mathcal{P}$ assigns each percentile of the function's resource‑demand 
distribution to a specific node. This percentile‑based splitting is illustrated 
in Fig.~\ref{fig:placement_p}.

\textbf{Step 1: Initialization \& Demand Normalization.}
The algorithm maintains two key data structures initialized in this step: i) a 
pending assignment vector $\mathbf{R}$, where $R[i]$ denotes the fraction of 
$F_i$ demand yet to be placed, initialized to $1$ for all functions; and ii) an 
assignment matrix $\mathbf{L}$, where $L[j,i]$ records the fraction of $F_i$ 
demand currently assigned to node $N_j$, initialized to all zeros (line~2). 
Furthermore, we scales the demand vector $\mathbf{V}_i$ of each function 
proportionally so that the aggregate demand matches the total cluster capacity 
($\sum_i \mathbf{V}_i = \sum_j \mathbf{C}_j$). This normalization ensures the 
subsequent placement operates under a balanced supply-demand premise (Line~3$\sim$4). 

\textbf{Step 2: Iterative Optimization Loop.}
The core of the algorithm is an iterative loop that runs until all function demands 
are placed ($R[i]=0, \forall i$, Line~7). In each iteration, it randomly selects 
a function $F_a$ with pending demand ($R[a] > 0$, Line~8). For each node $N_j$, 
the algorithm evaluates two ways to accommodate $F_a$: i) using the idle node 
resources, which yields a capacity $\alpha_j^{\text{free}}$ computed as the 
minimum ratio of idle CPU/memory to $F_a$ demand (Line~11); and ii) through 
beneficial swaps, where the subroutine $\text{SwapCapacity}$ attempts to evict 
already-placed functions in order to host as much of $F_a$ demand as possible, 
provided that doing so improves the node utilization $U_j$ (Line~13, detailed 
in Algorithm~\ref{alg:swap-capacity}). $\text{SwapCapacity}$ returns both the 
maximum $F_a$ demand fraction $\alpha_j^{\text{swap}}$ that can be hosted by 
function swaping, and a list $\mathcal{E}_j$ of evicted functions, each with the 
evicted load fraction $\beta$. The total accommodation capacity for $F_a$ on 
$N_j$ is then $\alpha_j^{\text{max}} = \min(\alpha_j^{\text{free}}+\alpha_j^{\text{swap}}, R[a])$ 
(Line~14).

After evaluating all nodes, the algorithm sorts them in descending order of the 
utilization gain $\Delta U_j$ (Line~17) and assigns $F_a$ pending demand greedily. 
For each node in this order, it assigns up to $\alpha_j^{\text{max}}$ of $F_a$ demand, 
updates $\mathbf{L}$ and $\mathbf{R}$ (Line~19), and simultaneously processes 
the evictions in $\mathcal{E}_j$: for each function $F_k$ in $\mathcal{E}_j$, 
the algorithm reducing the evicted fraction $\beta_k$ in $\mathbf{L}[j, k]$ and 
adds $\beta_k$ back to $F_k$ pending demand $\mathbf{R}[k]$ (Line~21). This process 
continues until $F_a$ demand is fully placed ($R[a]=0$). Because the total 
function demand was scaled to match the total cluster capacity in Step 1, it is 
guaranteed that $F_a$ demand can eventually be fully accommodated, and the loop 
will not terminate with $R[a] > 0$. To ensure the loop convergence speed, the 
algorithm skips nodes whose utilization gain $\Delta U_j$ is below a predefined 
empirical threshold (\emph{e.g.}, 1\%).

{
\begin{figure}[t]
  \centering
  \includegraphics[width=\linewidth]{figures/design/placement_p.pdf}
  \vspace{-10pt}
  \caption{Generate $\mathcal{P}$ from $L$,.}
  \vspace{-10pt}
  \label{fig:placement_p}
\end{figure}
}

\textbf{Step 3: Fine-Grained Placement Generation.}
After the assignment matrix $\mathbf{L}$ is determined, the algorithm converts 
the proportional assignments into a precise placement mapping $\mathcal{P}$ 
(Line~24$\sim$29). This process is illustrated in Fig.~\ref{fig:placement_p}. 
For each function $F_i$, the algorithm sequentially allocates contiguous percentile 
intervals in $[0, 1]$ to each node $N_j$ that hosts $F_i$ (left side of the figure). 
The length of the interval assigned to $N_j$ is exactly $L[j,i]$. This completes 
the construction of $\mathcal{P}$, which is the algorithm’s final output. In 
practice, to avoid resource fragmentation that could arise if certain nodes only 
receive large tasks, the intervals are assigned to nodes in a randomized order 
rather than a fixed sequence; the figure omits this step for simplicity.

\parabf{Task Dispatching.} At runtime, the interval $[0, 1]$ is concretely mapped 
to the resource‑demand percentiles of $F_i$’s invocations (right side of 
Fig.~\ref{fig:placement_p}). Specifically, using $F_i$’s historical demand 
distribution $D_i$, we pre‑compute the resource‑demand boundaries that correspond 
to each percentile sub‑interval. When a new invocation arrives, its actual resource 
demand (derived from input parameters and data size) is located on $D_i$ to obtain 
its percentile $p$. The invocation is then dispatched to the node $N_j$ for which 
$\mathcal{P}(F_i, p) = N_j$. For example, the newly arrived invocation marked in 
the figure, whose demand percentile falls inside $[0.8, 1]$, would be dispatched 
to Node C. Our experiments obtain accurate demand estimates from the benchmark’s 
known input characteristics; the algorithm can be coupled with more sophisticated 
prediction methods for further performance gains.

\begin{algorithm}
\caption{SwapCapacity: Compute beneficial swap capacity}
\label{alg:swap-capacity}
\begin{algorithmic}[1]
\Require $ $
\Statex \begin{itemize}
    \setlength{\itemindent}{.0em} % 控制缩进
    \item Node $N_j$ with idle resource vector $\mathbf{C}_j^{\text{idle}}$
    \item Target function $F_a$
    \item Allocation matrix $\mathbf{L}$
\end{itemize}
\Ensure Swap capacity $\alpha^{\text{swap}}$, eviction list $\mathcal{E} = [ (F_k, \beta_k) ]$
\State $\alpha^{\text{swap}} \gets 0$, $\mathcal{E} \gets [\,]$, $\{F_k\} \gets \{F_k | L[j, k] > 0\}$
\If{$C_j^{\text{idle,cpu}} > 0$} \Comment Determine dominant idle resource
    \State // Replace more memory‑intensive first
    \State sort $\{F_k\}$ by $r_k$ \textbf{ascending} ; $\text{sign} \gets +1$
\Else \Comment{ $C_j^{\text{idle,mem}} > 0$ }
    \State // Replace more compute‑intensive first
    \State sort $\{F_k\}$ by $r_k$ \textbf{descending}; $\text{sign} \gets -1$
\EndIf
\For{each $F_k$ in the sorted order}
    \If{$\text{sign} \cdot (r_a - r_k) \leq 0$} 
        \State \textbf{continue} \Comment{Swap would not improve $U(N_j)$}
    \EndIf
    \State Compute max $\beta_k$ s.t. evicting $\beta_k$ of $F_k$ respects $\mathbf{C}_j^{\text{idle}}$
    \State $\beta_k \gets \min(\beta_k,\; L[j, k])$
    \State $\gamma_k \gets \beta_k \cdot (\mathbf{V}_k \oslash \mathbf{V}_a)$ 
        \Comment{$\gamma_k = \beta_k \cdot \min\left(\frac{V_k^{\text{cpu}}}{V_a^{\text{cpu}}}, \frac{V_k^{\text{mem}}}{V_a^{\text{mem}}}\right)$}
    \If{$\alpha^{\text{swap}} + \gamma_k > R[a]$}
        \State $\gamma_k \gets R[a] - \alpha^{\text{swap}}$; adjust $\beta_k$ proportionally
    \EndIf
    \State $\alpha^{\text{swap}} \gets \alpha^{\text{swap}} + \gamma_k$
    \State $\mathcal{E} \gets \mathcal{E} \;\cup\; \{ (F_k, \beta_k) \}$
    \If{$\alpha^{\text{swap}} \geq R[a]$} \textbf{break} \EndIf
\EndFor
\State \Return $(\alpha^{\text{swap}},\; \mathcal{E})$
\end{algorithmic}
\end{algorithm}

\parabf{Assignment via beneficial swaps.}
The SwapCapacity subroutine (Algorithm~\ref{alg:swap-capacity}) determines, for 
a given node $N_j$ and a target function $F_a$, the maximum additional fraction 
$\alpha_j^{\text{swap}}$ of $F_a$ demand that can be hosted by beneficially evicting 
already-placed functions. A swap is considered beneficial only if it increases 
the overall node utilization $U(N_j)$. This requirement dictates that swaps must 
consume the node’s dominant idle resource: if CPU is idle, $F_a$ must replace 
more memory-intensive functions (with smaller demand ratio $r_k$); if memory is 
idle, it must replace more compute-intensive ones (with larger $r_k$). The subroutine 
takes as input the node idle resource vector $\mathbf{C}_j^{\text{idle}}$, the 
target function $F_a$, and the current assignment matrix $\mathbf{L}$; it returns 
the admissible swap capacity $\alpha^{\text{swap}}$ together with a list $\mathcal{E}$ 
of evicted functions, each annotated with the evicted load fraction $\beta_k$.

The subroutine proceeds in four logical steps. First, it identifies the dominant 
idle resource—CPU or memory—since after using the node’s free capacity (line~11 
of Algorithm 1), only one resource dimension may remain idle (lines 2$\sim$7). 
This determines the order in which existing functions are examined: if CPU is 
idle, functions are sorted by their resource-demand ratio $r_k$ in ascending 
order (\emph{i.e.}, more memory‑intensive functions first); if memory is idle, 
they are sorted in descending order (more compute‑intensive first). This ordering 
ensures that the most beneficial candidates are tried first, because replacing 
a function with a very complementary resource profile allows $F_a$ to consume 
more of the idle resource.

Second, for each candidate function $F_k$ in this sorted list, the algorithm 
performs a benefit test (line~9). The condition $\operatorname{sign}\cdot(r_a-r_k) > 0$ 
guarantees that swapping a fraction of $F_k$ for $F_a$ will actually increase 
$U(N_j)$. Intuitively, if CPU is idle ($\operatorname{sign}=+1$), we must have 
$r_a > r_k$, meaning $F_a$ is more compute‑intensive than $F_k$ and will better 
utilize the spare CPU; the converse holds when memory is idle. Candidates that 
fail this test are skipped.

Third, for a candidate that passes the test, the algorithm computes how much of $F_k$ can be evicted ($\beta_k$) without exceeding the node’s idle resources, and limits $\beta_k$ by the amount currently assigned to $N_j$ ($L[j,k]$) (lines~11$\sim$12). It then translates this evicted capacity into an equivalent amount of $F_a$ that can be hosted using the freed resources. The translation uses the formula $\gamma_k = \beta_k \cdot \min\left(\frac{V_k^{\text{cpu}}}{V_a^{\text{cpu}}}, \frac{V_k^{\text{mem}}}{V_a^{\text{mem}}}\right)$(line~13), which ensures that neither CPU nor memory exceeds the vacated capacity—the $\min$ operation enforces the tighter of the two resource constraints.

Finally, the algorithm accumulates the admissible fractions $\gamma_k$ until 
either the pending demand $R[a]$ of $F_a$ is exhausted or no further beneficial 
swaps can be found (lines~14$\sim$17). The accumulated total $\alpha^{\text{swap}}$ 
and the corresponding eviction list $\mathcal{E}$ are returned to the main placement loop.

\parabf{Low Overhead via Partial Updates.} To keep the scheduling overhead low, 
the placement algorithm employs an incremental (partial) update mechanism. Rather 
than recomputing the full placement from scratch for every scheduling window, the 
system periodically (every 5 minutes in our implementation) monitors the latest 
demand profiles $\langle r_i, \mathbf{V}_i, D_i \rangle$ of all functions. For 
a function whose aggregate demand $\mathbf{V}_i$ has not increased and demand 
distribution $D_i$ remains statistically stable, the algorithm preserves its 
current placement. Only functions with significantly changed demand (increased 
$\mathbf{V}_i$ or altered $D_i$) are fed into the complementary swarming algorithm 
to update their assignments. This partial update drastically reduces the number 
of functions that need to be re‑placed in each iteration. Furthermore, to prevent 
the accumulation of sub‑optimality that may arise from multiple incremental 
updates, the system performs a global re‑placement every 30 minutes, which 
recomputes the assignment for all functions from a clean state, ensuring long‑term 
placement quality.

\subsection{Monotonic Scale-up Task Scheduling}
\label{sec:design:scheduling}

% 1. 调度器目标与总览 (Scheduler Objective & Overview)
%   - 重申目标:解决函数内需求波动,实现无重启细粒度资源调整。
%   - 总览架构图:展示双队列、多沙箱、与K8s运行时交互的组件图。

% 2. 调度器核心结构 (Core Scheduler Structure)
%   - **双任务队列**:明确两个队列的作用(例如,Q1: 等待分配沙箱的任务;Q2: 已分配待执行的任务)。
%   - **排序规则**:队列按预测资源需求单调递增排序。
%   - **多沙箱均分优先级结构**:如何将不同函数的请求分配到不同沙箱,以及如何在沙箱间平衡负载。

% 3. 优先级设计 (Priority Design)
%   - 定义优先级函数,综合考虑等待时间、冷启动开销、资源需求。
%   - 解释如何通过“在优先级约束下策略性重排序”来延长队列,从而提升利用率和减少冷启动。

% 4. 事件驱动调度逻辑 (Event-driven Scheduling Logic)
%   - 列出关键事件:新请求到达、沙箱资源调整完成、沙箱空闲、资源回收触发。
%   - 描述调度器对每个事件的反应逻辑(可配流程图或状态机)。
\section{Implementation}
\label{sec:implementation}

We implement a prototype of \sysname with $\sim$3100 LOC in Python, with 
Kubernetes~\cite{k8s} for instance management, Minio~\cite{MinIO} for function 
data management, and SciPy~\cite{scipy} for workload prediction.

\parabf{Instance manager.} We use containers as the serverless function instance 
based on the Kubernetes container orchestration. Each function instance is exposed 
via a unique Pod IP address, which the scheduler maintains in a pod-IP mapping 
table. Container resource limits and instance placement are enforced through 
Kubernetes Pod specifications and node-affinity rules, respectively. In the 
prototype, the scheduler must manage a large number of instances and frequently 
interact with each one to perform lifecycle operations—including startup, access, 
scaling, termination, and health monitoring. To reduce control-plane overhead, 
our implementation leverages the Python Kubernetes client library in a multi-process 
and multi-coroutine architecture. Since some benchmarks require reading from and 
writing to external data during execution, we additionally deploy a MinIO service 
within the cluster to provide a unified data management layer for the function 
system.

\parabf{Scheduler.} The scheduler in the prototype consists of four modules.
i) The dispatching plan generator runs as a periodic cluster-level service (currently every minute). It executes the complementary swarming dispatching algorithm to compute the optimized request dispatching plan, which is provided to the request dispatcher.
ii) The request dispatcher operates as a separate cluster-level process. It assigns each incoming request to a worker node according to the current dispatching plan received from the generator.
iii) The worker scheduler runs locally on each worker node. It maintains per-function request queues, enforces the monotonic scale-up scheduling policy to determine execution order, and triggers in-place pod scaling when required.
iv) The system monitor collects real-time resource usage and per-function metrics (e.g., cold-start time, execution duration, resource demand distribution). For load prediction, it currently employs a basic FFT-based method implemented with the Python SciPy~library~\cite{scipy}.

\section{Evaluation}
\label{sec:experiments}

In this section, we first evaluate the overall performance of \sysname against 
state-of-the-art systems—Palette~\cite{abdi2023palette} and 
FaaSCache~\cite{ASPLOS21:FaaSCache}—in \S\ref{sec:experiments_comparsion}. Next, 
we investigate the scalability of \sysname in large-scale cluster environments 
in \S\ref{sec:experiments_micro}. Finally, we conduct an in-depth analysis of 
the effectiveness of the key techniques employed in \sysname in \S\ref{sec:experiments_technique}.

\subsection{Evaluation Setup}
\label{sec:experiments_setup}

\parabf{Benchmarks.}
We selected ten benchmark applications from a recent open-source serverless 
benchmarking suite~\cite{copik2021sebs} to evaluate the performance of \sysname 
across diverse workloads. Table\ref{tab:setup} summarizes the basic distribution 
of these benchmarks (see~\cite{xxx} for full details). The selected benchmarks 
span multiple categories—including Web Apps, Multimedia, Utilities, Inference, 
and Scientific Computing—and exhibit substantial variation in resource requirements. 
In our testdeb, the average memory consumption of these benchmarks ranges from 
500 MB to 4 GB, while CPU usage varies between 0.5 and 4 cores. The memory-to-CPU 
ratio (GB per core) also differs significantly, spanning from 0.5 to 4. Moreover, 
the average execution times range from 0.15 to 21.5 seconds for these applications. 
This diversity in workload characteristics enables a comprehensive performance 
evaluation of \sysname under realistic conditions.

\begin{table}[ht]
    \vspace{-10pt}
    \caption{Performance comparison under multiple workflows.\label{tab:setup}}
    % \fontsize{20pt}{30pt}\selectfont
    \begin{tabularx}{0.47\textwidth}{ 
        >{\raggedright\arraybackslash}X  % 1. 第一列左对齐
        >{\centering\arraybackslash}X    % 2. 其他列居中...
        >{\centering\arraybackslash}p{5cm}} % 3. 第四列指定为更宽的2cm,并居中
        \toprule
        \textbf{Benchmark} & \textbf{Type} & \textbf{Distribution} \\
        \hline 
        \hline 
        \raisebox{-5pt}{graph-bfs} & \raisebox{-5pt}{Scientific} & Breadth-first search traversal implemented using igraph. \\
        \hline
        graph-mst & \raisebox{-5pt}{Scientific} & Computing minimum spanning tree using igraph. \\
        \hline
        graph-pagerank & \raisebox{-5pt}{Scientific} & \raisebox{-5pt}{PageRank algorithm execution with igraph.} \\
        \hline
        dna-visualisation & \raisebox{-5pt}{Scientific} & Generating visualization data from DNA sequences. \\
        \hline
        dynamic-html & \raisebox{-5pt}{Webapps} & Rendering dynamic HTML content from a template. \\
        \hline
        \raisebox{-5pt}{uploader} & \raisebox{-5pt}{Webapps} & Uploading files from a given URL to cloud storage. \\
        \hline
        \raisebox{-5pt}{compression} & \raisebox{-5pt}{Utilities} & Compressing multiple files into a ZIP archive for download. \\
        \hline
        image-recognition & \raisebox{-5pt}{Inference} & Performing image classification using ResNet and PyTorch. \\
        \hline
        thumbnailer & Multimedia & Creating thumbnail images from input pictures. \\
        \hline
        video-processing & \raisebox{-5pt}{Multimedia} & Adding watermarks and generates GIFs from video files. \\
        \bottomrule
    \end{tabularx}
\end{table}

\parabf{Workload generation.} We generate simulated user traffic based on the 
Azure Functions Dataset~\cite{ShahradFGCBCLTR20}. For each benchmark, a 
corresponding function invocation trace is randomly sampled from the top 20\% 
most frequently invoked functions—a subset reported to account for the majority 
of resource consumption in production environments~\cite{ShahradFGCBCLTR20}, 
thus representing the critical bottleneck for cluster performance optimization. 
The request rate of each trace is proportionally scaled so that the aggregated 
resource demand approaches the total capacity of the cluster, while the invocation 
frequencies across the ten benchmarks are adjusted to be nearly identical (with a 
relative difference of less than 2\%). Since the original trace data is recorded 
at minute-level granularity, we model intra-minute request arrivals using a 
Poisson process. In alignment with common industrial practice~\cite{aws_scaling_window, 
knative_scaling_window}, the scaling interval is set to 1 minute.

To simulate the real-world characteristic where the resource demand of a serverless 
function follows a normal distribution across multiple executions, we first 
randomly select ten functions from the dataset and compute the variance of their 
resource usage after Gaussian fitting. We then randomly adjust the input data 
size and parameters of each benchmark so that the actual resource requirements 
align with the sampled variance. Given that the dataset only reflects the distribution 
of memory resource demands, we assume a fixed ratio between CPU and memory requirements 
for each benchmark. Accordingly, CPU allocation is determined proportionally 
based on the memory~demand.

\parabf{Cluster configuration.} Our experiments are conducted on CloudLab using 
a cluster consisting of 9 c6620 instances (56 cores, 2 * Intel Xeon Gold 5512U 
@ 2.1GHz, 128GB memory). Eight of them serve as working servers and one as the 
scheduler and user request generator. To simulate the varied resource availability 
conditions in production clusters, our evaluation allocates only a subset of total 
resources on each worker server. Unless otherwise specified, the available resource 
proportions across different servers are sampled from a normal distribution with 
a standard deviation of 0.85.

\parabf{Baseline.} We evaluate \sysname against two state-of-the-art resource 
management approaches for serverless platforms: Palette~\cite{abdi2023palette} 
and FaaSCache~\cite{ASPLOS21:FaaSCache}, which optimize cluster efficiency from 
complementary perspectives. \textbf{Palette} enhances spatial locality by allowing 
users to assign color labels to functions and co-locating those with the same color 
on the same server. In our experiments, each benchmark is assigned a unique color. 
Inspired by caching systems, \textbf{FaaSCache} employs a Greedy-Dual keep-alive 
strategy that assigns a priority to each kept-alive instance and evicts low-priority 
instances first under resource pressure. To ensure a fair comparison and minimize 
the impact of engineering artifacts, we reimplement both baselines in our prototype 
environment. Furthermore, to better reflect real-world production settings while 
maintaining controlled conditions, we enable Kubernetes vertical pod autoscaling 
(VPA) across all systems during evaluation (see prototype implementation details 
in Section~\ref{sec:implementation}).

\parabf{Metric.} We evaluate system performance using memory utilization, CPU 
usage, and task throughput as primary metrics. To verify that the scheduling 
algorithm in \sysname does not lead to task starvation, we also compare the 95th 
percentile tail latency of task processing. Each data point is collected from 
one-hour-long test runs, with the performance data from the first five minutes 
excluded to eliminate the impact of initial cold starts.

\subsection{Benefits of \sysname}
\label{sec:experiments_comparsion}

\begin{figure*}[ht]
    \centering
    \subfloat[]{\includegraphics[width=0.32\linewidth]{figures/Eval/distribution_CPU Utilization.pdf}%
    \label{fig:eval_dis_cpu}}
    \hfil
    \subfloat[]{\includegraphics[width=0.32\linewidth]{figures/Eval/16worker_distribution_Memory Utilization.pdf}%
    \label{fig:eval_dis_memory}}
    \hfil
    \subfloat[]{\includegraphics[width=0.32\linewidth]{figures/Eval/distribution_Task throughput.pdf}%
    \label{fig:eval_dis_throughput}}
    \caption{Performance comparison under different cluster resource distribution. (a) CPU Utilization. (b) Memory Utilization. (c) Task throughput.}
    \vspace{-15pt}
\label{fig:eval_dis}
\end{figure*}

To better demonstrate the performance of \sysname under different deployment 
scenarios, we first compare the overall performance among \sysname and baselines 
under different cluster resource distribution, benchmark number and workload distributions.

\parabf{Comparison under different cluster resource distribution.} Figure~\ref{fig:eval_dis} 
compares the performance of \sysname and baselines under different cluster resource 
distribution. We introduce two resource availability patterns: uniform 
and Zipf distributions. Under the uniform distribution, each server is configured 
with 75\% of its total resources available. In the Zipf-based distribution, the 
fraction of available resources per server follows a Zipf distribution parameterized 
with $\alpha = 0.8$ across $n = 8$ resource tiers. 

Across all resource distribution patterns, \sysname consistently improves both 
CPU and memory utilization, achieving relative improvements of 43.61\%$\sim$55.53\% 
and 41.41\%$\sim$67.01\% over the baselines, respectively. Furthermore, \sysname 
demonstrates performance gains in task throughput, with improvements ranging from 
51.52\% to 103.71\%.

To address the potential concern of task starvation, we examine the impact on 
request-level performance. As shown in Table~\ref{tab:latency}, \sysname reduces 
the overall 95th percentile latency by 30.63\%$\sim$42.42\% compared to the baselines. 
This result indicates that the efficiency gains of \sysname are not achieved by 
systematically delaying particular requests or task types. Instead, the improvement 
in resource utilization and throughput is achieved alongside enhanced and more 
predictable end-to-end performance, even when co-locating diverse benchmarks with 
varying resource demands and execution characteristics.

The performance gains of \sysname are primarily attributed to two mechanisms. 
First, the workload-splitting-based instance placement strategy enables individual 
servers to serve fewer functions, thereby increasing instance reuse probability. 
This approach also aligns server workload with resource capacity, avoiding resource 
waste caused by load-resource mismatch. Second, the vertical-scaling-based task 
scheduling policy reduces both resource over-provisioning and cold-start frequency, 
further boosting overall resource utilization.

\begin{table}[ht]
    \vspace{-10pt}
    \caption{Performance comparison under multiple workflows.\label{tab:latency}}
    \begin{tabularx}{0.47\textwidth}{ 
        >{\raggedright\arraybackslash}X 
        >{\centering\arraybackslash}X
        >{\centering\arraybackslash}X
        >{\centering\arraybackslash}X }
        \toprule
        \multirow{2}{*}{\makecell[l]{\textbf{Resource} \\ \textbf{Distribution}}} & \multicolumn{3}{c}{\textbf{95th Tail Latency (s)}} \\ \cline{2-4}
         & \raisebox{-1pt}{\textbf{Palette}} & \raisebox{-1pt}{\textbf{FaaSCache}} & \raisebox{-1pt}{\textbf{\sysname}} \\
        \hline 
        Norm-85 & 1071 & 1104 & \textbf{743} \\
        Uniform-75\%& 1054 & 1063 & \textbf{729} \\
        Zipf-80 & 1372 & 1335 & \textbf{790} \\
        \bottomrule
    \end{tabularx}
\end{table}

\begin{figure*}[ht]
    \centering
    \subfloat[]{\includegraphics[width=0.32\linewidth]{figures/Eval/func_number_CPU Utilization.pdf}%
    \label{fig:eval_function_cpu}}
    \hfil
    \subfloat[]{\includegraphics[width=0.32\linewidth]{figures/Eval/func_number_Memory Utilization.pdf}%
    \label{fig:eval_function_memory}}
    \hfil
    \subfloat[]{\includegraphics[width=0.32\linewidth]{figures/Eval/func_number_Task throughput.pdf}%
    \label{fig:eval_function_throughput}}
    \caption{Performance comparison under different function numbers. (a) CPU Utilization. (b) Memory Utilization. (c) Task throughput.}
    \vspace{-15pt}
\label{fig:eval_function}
\end{figure*}

\parabf{Comparison under different benchmark number.}
Next, we evaluate \sysname and the baselines under different workload scales. 
Beyond the default configuration of 10 benchmarks, we provide two additional scenarios:

\begin{itemize}
    \item \textbf{5 benchmarks}: A randomly selected subset comprising \textbf{video-processing}, 
    \textbf{graph-pagerank}, \textbf{image-recognition}, \textbf{dna-visualization}, 
    and \textbf{thumbnailer}. To maintain the comparable total load, the 
    invocation rate for each function in this set is doubled;

    \item \textbf{20 benchmarks}: The original 10 benchmarks plus their replicas. Each 
    replica has the same code, resource profile, and runtime as the original but 
    is considered a distinct function, preventing instance reuse between the 
    original-replica pairs. The invocation rate per function is halved to 
    maintain the aggregate request rate.
\end{itemize}

As shown in Figure~\ref{fig:eval_function}, the performance advantage of \sysname 
over the baselines grows as the number of benchmarks increases from 5 to 20. 
Specifically, the improvement in CPU utilization rises from 17.40\%$\sim$16.90\% 
to 44.89\%$\sim$44.06\%, memory utilization gains increase from 41.81\%$\sim$43.26\% 
to 55.42\%$\sim$57.37\%, and task throughput improvements expand from 45.22\%$\sim$51.06\% 
to 75.45\%$\sim$78.98\%.

This scaling trend can be attributed to the workload-splitting strategy. With 
fewer benchmarks, each server already operates on a limited set of functions, 
leaving limited room for further optimization through instance reuse. In contrast, 
when the workload is larger and more diverse, the strategy reduces the number of 
distinct functions each server handles. This effect elevates the instance reuse 
rate, which in turn improves overall~efficiency.

\begin{figure*}[ht]
    \centering
    \subfloat[]{\includegraphics[width=0.32\linewidth]{figures/Eval/skew_CPU Utilization.pdf}%
    \label{fig:eval_workload_cpu}}
    \hfil
    \subfloat[]{\includegraphics[width=0.32\linewidth]{figures/Eval/skew_Memory Utilization.pdf}%
    \label{fig:eval_workload_memory}}
    \hfil
    \subfloat[]{\includegraphics[width=0.32\linewidth]{figures/Eval/skew_Task throughput.pdf}%
    \label{fig:eval_workload_throughput}}
    \caption{Performance comparison under different workload distributions. (a) CPU Utilization. (b) Memory Utilization. (c) Task throughput.}
    \vspace{-15pt}
\label{fig:eval_workload}
\end{figure*}

\parabf{Comparison under different workload distributions.} 
Finally, we examine the impact of workload distribution on scheduling performance. 
In addition to the default resource distribution (see~\ref{sec:experiments_setup}), 
we introduce a Zipf-based load pattern. Specifically, we adjust the $\alpha$ parameter 
of a Zipf distribution (with n = 10) to obtain a probability set that approximates 
the Pareto principle (\emph{i.e.}, the 80/20 rule), where the sum of the two largest 
probabilities is about 0.8. We then scale the request rate of each benchmark 
according to this distribution, preserving the total aggregate invocation frequency 
while ensuring that the two most frequently invoked functions—randomly chosen as 
\textbf{thumbnailer} and \textbf{video-processing}—collectively account for 80\% 
of all invocations.

It can be observed from Figure~\ref{fig:eval_workload} that, compared to the default 
uniform distribution, the optimization margin of \sysname under the Zipf workload 
distribution is relatively smaller. Under the Zipf distribution, \sysname achieves 
an increase in CPU utilization of 23.20\%$\sim$23.58\%, memory utilization of 
27.04\%$\sim$28.50\%, and task throughput of 29.20\%$\sim$31.34\% over the baselines.

\looseness=-1
This result stems from our experimental configuration: the Zipf distribution 
adhering to the 80/20 rule implies that only two functions bear the majority of 
the workload, which naturally grants them more instance reuse opportunities. In 
contrast, while the real-world workload distribution also exhibits skew, it 
involves a larger number of high-load functions competing for server resources 
simultaneously. This competition creates more optimization potential for the 
\sysname strategy. Moreover, even with only two high-load functions, \sysname 
still achieves around 20\%$\sim$30\% higher resource utilization and task throughput 
than the baselines. This gain is primarily because the vertical-scaling-based 
task scheduling policy reduces both instance resource over-provisioning and 
cold-start frequency through task reordering, thereby enhancing overall 
resource~efficiency.

\begin{figure}[t]
    \centering
    \subfloat[]{\includegraphics[width=0.9\linewidth]{figures/Eval/detail_cpu.pdf}%
    \label{fig:eval_detail_cpu}}
    \hfil
    \subfloat[]{\includegraphics[width=0.9\linewidth]{figures/Eval/detail_memory.pdf}%
    \label{fig:eval_detail_memory}}
    \caption{Detail Resource Comsupmtion of \sysname and baselines. (a) CPU Detail Comsupmtion. (b) Memory Detail Comsupmtion.}
    \vspace{-15pt}
\label{fig:eval_detail}
\end{figure}

\parabf{Performance analysis.} To further demonstrate the reasons for \sysname 
performance advantage, Figure~\ref{fig:eval_detail} provides a detailed 
illustration of the average resource consumption of \sysname and the baselines 
during task processing. We categorize CPU and memory resources into the following 
four types: 

\begin{itemize}
    \item \textbf{Task Execution}: resources allocated for processing tasks;
    \item \textbf{Cold Start}: resources consumed during the initialization of instances;
    \item \textbf{Standby Instance}: resources held by ready but currently idle instances;
    \item \textbf{Idle Resource}: available but unallocated resources.
\end{itemize}

Taking CPU utilization as an example, compared to the baselines, \sysname increases 
the resource share for Task Execution by 45.47\%$\sim$43.61\%, for Standby Instance 
by 110.87\%$\sim$125.58\%, and for Idle Resource by 43.59\%$\sim$47.37\%, while 
it reduces the share for Cold Start by 83.65\%$\sim$83.56\%. A similar trend is 
observed for memory utilization.

The resource utilization advantage of \sysname stems from the reduction in cold 
starts. This is achieved through two strategies. First, the workload-splitting-based 
instance placement concentrates a narrower set of functions on each server, which 
increases instance reuse opportunities. Second, the vertical-scaling-based task 
scheduling policy reorders tasks and dynamically adjusts instance resources, 
enabling instances to serve varying demands throughout their lifecycle without 
repeated cold starts or static over-provisioning.

The resources conserved from fewer cold starts are reallocated in two ways: they 
increase the share for task execution while expanding the pool of standby instances. 
Consequently, a larger ready-to-use instance pool is maintained, raising the 
probability of warm starts.

\subsection{Effectiveness of \sysname}
\label{sec:experiments_technique}

{
\begin{figure}[t]
    \centering
    \includegraphics[width=0.97\linewidth]{figures/Eval/workflow_xiaorong.pdf}
    \caption{Performance comparison between \sysname and simplified versions.}
    \label{fig:technique}
    \vspace{-10pt}
\end{figure}
}

In this section, we investigate the impact of the techniques employed by 
\sysname. We measure \sysname with the following two simplified versions:
\begin{itemize}
    \item \textbf{Placement*}: This variant uses only the workload-splitting-based 
    instance placement strategy, while tasks within each server are scheduled using 
    a simple FIFO policy.
    \item \textbf{Scheduling*}: This variant uses only the vertical-scaling-based 
    task scheduling policy within each server, while user requests are randomly 
    assigned across servers.
\end{itemize}

Figure~\ref{fig:technique} shows the performance comparison between \sysname and 
simplified versions. We also provide the evaluation results of the baselines for 
reference.

As shown, the performance of the Placement* version lies between the baselines 
and \sysname. Compared to the baselines, it achieves improvements of 20.07\%$\sim$21.63\%, 
16.90\%$\sim$17.23\%, and 17.31\%$\sim$15.74\% in CPU utilization, memory utilization, 
and task throughput, respectively. However, it still falls short of \sysname by 
16.39\%, 17.33\%, and 23.62\% in the metrics. These results demonstrate that the 
workload-splitting-based instance placement strategy alone can effectively enhance 
scheduling performance through load partitioning. Furthermore, it requires integration 
with the vertical-scaling-based task scheduling policy for fine-grained, 
server-internal resource management to achieve optimal performance.

The performance of the Scheduling* variant reveals a more nuanced trade-off. It 
underperforms \sysname in resource utilization, with CPU and memory utilization 
lower by 13.35\% and 29.99\%, respectively. Despite this, it achieves a 27.95\% 
higher task throughput. This discrepancy occurs because Scheduling* prioritizes 
tasks with lower resource demands, shorter execution times, and higher invocation 
frequencies, without explicitly optimizing for overall server utilization.

When integrated with the workload-splitting-based instance placement strategy 
(as in \sysname), each server handles a limited, resource-complementary set of 
functions. This increases the likelihood that tasks with high resource demands 
are selected for instance setup and execution, raising the execution priority for 
subsequent tasks from the same function (see Section~\ref{sec:scheduling}). In 
contrast, without effective load partitioning, a wider variety of instances compete 
for server resources. In this scenario, resource-intensive instances are served 
mainly through starvation-avoidance mechanisms, receiving only limited execution 
opportunities. The resulting contention leads to frequent instance eviction and 
cold starts, which hinders effective improvement in resource utilization.

\subsection{Scalability}
\label{sec:experiments_overhead}

Finally, we evaluate the scalability of \sysname. Due to testbed limitations, 
this section includes both real and simulated experiments. We first demonstrate 
the performance of \sysname on a larger-scale cluster. Then we
show the system overhead of the \sysname scheduler on up to 1024 nodes.

\begin{figure*}[ht]
    \centering
    \subfloat[]{\includegraphics[width=0.32\linewidth]{figures/Eval/16worker_distribution_CPU Utilization.pdf}
    \label{fig:eval_big_cpu}}
    \hfil
    \subfloat[]{\includegraphics[width=0.32\linewidth]{figures/Eval/16worker_distribution_Memory Utilization.pdf}
    \label{fig:eval_big_memory}}
    \hfil
    \subfloat[]{\includegraphics[width=0.32\linewidth]{figures/Eval/16worker_distribution_Task throughput.pdf}
    \label{fig:eval_big_throughput}}
    \caption{Performance comparison in 16-worker cluster. (a) CPU Utilization. (b) Memory Utilization. (c) Task throughput.}
    \vspace{-15pt}
\label{fig:eval_big}
\end{figure*}

\parabf{Larger-scale cluster experiment.}
In the larger-scale cluster experiment, we increased the number of worker nodes 
to 16 and proportionally scaled the request invocation frequency, while keeping 
other conditions constant. Figure~\ref{fig:eval_big} shows the performance comparison 
among \sysname and the baselines. Compared to the baselines, \sysname improves 
CPU utilization by 31.80\%$\sim$61.75\%, memory utilization by 41.95\%$\sim$53.82\%, 
and task throughput by 40.24\%--110.75\%. These results align with those observed 
in the smaller cluster, demonstrating that \sysname maintains its performance 
advantage stably at scale.

{
\begin{figure}[t]
    \centering
    \includegraphics[width=0.97\linewidth]{figures/Eval/overhead_placement.pdf}
    \caption{The average latency of \sysname in handling scaling request.}
    \label{fig:overhead_placement}
    \vspace{-10pt}
\end{figure}
}

{
\begin{figure}[t]
    \centering
    \includegraphics[width=0.97\linewidth]{figures/Eval/overhead_scheduling.pdf}
    \caption{The average latency of \sysname in handling scaling request.}
    \label{fig:overhead_scheduling}
    \vspace{-10pt}
\end{figure}
}

\parabf{System overhead.}
We next evaluate the overhead of the \sysname scheduler under varying cluster 
sizes and request frequencies. To do so, we simulate scheduler load for clusters 
ranging from 64 to 1024 worker servers. To avoid interference from bursty traffic 
patterns, we generate invocations at a steady rate instead of using the trace 
from the Azure Functions Dataset~\cite{ShahradFGCBCLTR20}. The experimental results 
are presented in Figures~\ref{fig:overhead_placement} and~\ref{fig:overhead_scheduling}. 

Figure~\ref{fig:overhead_placement} shows the average latency for generating cluster-wide 
instance placement strategies under different cluster sizes. It can be observed 
that the placement-generation overhead increases gradually with cluster scale but 
remains largely unaffected by request frequency. This is because \sysname scales 
the overall load proportionally to the cluster size before computing the placement 
strategy. Even with 1024 worker nodes and an invocation rate of $250 \times 10^3$ 
requests per minute, the latency stays below 110 ms. This result indicates that 
the scheduler can generate placement strategies promptly, even in large-scale 
clusters, demonstrating the practical scalability of \sysname.

Figure~\ref{fig:overhead_scheduling} shows the scheduling latency for individual 
task assignments under different invocation frequencies. It should be noted that 
placement strategy generation is performed by a separate background process within 
the scheduler. The main scheduling process only receives the resulting placement 
plan and assigns tasks accordingly. Therefore, only the task assignment overhead 
contributes to the additional latency introduced by \sysname in the end-to-end 
task execution path. As shown, even under a heavy load of 1024 worker nodes, the 
extra scheduling delay added by \sysname remains below 10 ms. This overhead is 
negligible for the vast majority of FaaS scenarios, confirming that the scheduling 
logic of \sysname is lightweight enough for production-scale deployment.

\section{Related Work}
\label{sec:related}

\paraf{Serverless runtime resource configuration.}


\parabf{Serverless autoscaling and instance placement.}


\parabf{Serverless request scheduling and dispatching.}

\section{Conclusion}
This paper presents \sysname, a FaaS scheduler designed to address the resource utilization challenges caused by heterogeneous function resource demands. First, \sysname co-locates functions at the cluster-level with complementary swarming instance placement strategy, which not only improves overall server utilization but also increases instance-reuse opportunities by narrowing the function serving set per server. Second, with monotonic scale-up task scheduling, \sysname aligns task processing ordering with the in-place scaling capability of Kubernetes pods. This allows pod resource allocation to be dynamically adjusted according to actual task demands, effectively reducing cold starts without over-provisioning. As demonstrated in the evaluation, \sysname shows significant improvements over the baselines in both resource efficiency and task throughput, while adding negligible system overhead and maintaining scalable performance. Flexible, fine-grained resource configuration is becoming a prevailing trend in modern FaaS platforms, which presents both new challenges and opportunities for scheduling. We hope that our design can serve as a reference and inspire further research in this direction.

\vspace{-7pt}
\label{lastpage}
% \clearpage


% \section*{Acknowledgments}
% This should be a simple paragraph before the References to thank those individuals and institutions who have supported your work on this article.

\bibliographystyle{myIEEEtran}
\bibliography{xin}

%\begin{IEEEbiography}[{\includegraphics[width=1in,height=1.25in,clip,keepaspectratio]{example-image}}]{Xuanzhe Liu}
%(Senior Member, IEEE) is a  Full {P}rofessor in the School of Computer Science {at} Peking University, {{ Beijing, China}. 
%His research interests mainly fall in service-based software engineering and systems.
%{Most of his recent efforts have been published at prestigious conferences including WWW, ICSE, FSE, SOSP, SIGCOMM, NSDI, MobiCom, MobiSys, and in journals including ACM TOSEM/TOIS and IEEE TSE/TMC/TSC. 
%He is a distinguished member of the ACM and the CCF. Web page: \url{http://www.liuxuanzhe.com/}}}.
%\end{IEEEbiography}
%
%\vspace{-15pt}

\begin{IEEEbiography}[{\includegraphics[width=1in,height=1.25in,clip,keepaspectratio]{example-image}}]{Xin Jin}
(Senior Member, IEEE) received the PhD degree from Princeton University, in 2016. He is currently an associate professor (with Tenure) with the School of Computer Science, Peking University. His research interests include computer systems, networking, and cloud computing.
\end{IEEEbiography}
  


\vfill
\label{allpages}


\end{document}


