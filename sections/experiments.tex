\section{Evaluation}
\label{sec:experiments}

In this section, we first evaluate the overall performance of \sysname against 
state-of-the-art systems—Palette~\cite{abdi2023palette} and 
FaaSCache~\cite{ASPLOS21:FaaSCache}—in \S\ref{sec:experiments_comparsion}. Next, 
we investigate the scalability of \sysname in large-scale cluster environments 
in \S\ref{sec:experiments_micro}. Finally, we conduct an in-depth analysis of 
the effectiveness of the key techniques employed in \sysname in \S\ref{sec:experiments_technique}.

\subsection{Evaluation Setup}
\label{sec:experiments_setup}

\parabf{Benchmarks.}
We selected ten benchmark applications from a recent open-source serverless 
benchmarking suite~\cite{copik2021sebs} to evaluate the performance of \sysname 
across diverse workloads. Table\ref{tab:setup} summarizes the basic distribution 
of these benchmarks (see~\cite{xxx} for full details). The selected benchmarks 
span multiple categories—including Web Apps, Multimedia, Utilities, Inference, 
and Scientific Computing—and exhibit substantial variation in resource requirements. 
In our testdeb, the average memory consumption of these benchmarks ranges from 
500 MB to 4 GB, while CPU usage varies between 0.5 and 4 cores. The memory-to-CPU 
ratio (GB per core) also differs significantly, spanning from 0.5 to 4. Moreover, 
the average execution times range from 0.15 to 21.5 seconds for these applications. 
This diversity in workload characteristics enables a comprehensive performance 
evaluation of \sysname under realistic conditions.

\begin{table}[ht]
    \vspace{-10pt}
    \caption{Performance comparison under multiple workflows.\label{tab:setup}}
    % \fontsize{20pt}{30pt}\selectfont
    \begin{tabularx}{0.47\textwidth}{ 
        >{\raggedright\arraybackslash}X  % 1. 第一列左对齐
        >{\centering\arraybackslash}X    % 2. 其他列居中...
        >{\centering\arraybackslash}p{5cm}} % 3. 第四列指定为更宽的2cm,并居中
        \toprule
        \textbf{Benchmark} & \textbf{Type} & \textbf{Distribution} \\
        \hline 
        \hline 
        \raisebox{-5pt}{graph-bfs} & \raisebox{-5pt}{Scientific} & Breadth-first search traversal implemented using igraph. \\
        \hline
        graph-mst & \raisebox{-5pt}{Scientific} & Computing minimum spanning tree using igraph. \\
        \hline
        graph-pagerank & \raisebox{-5pt}{Scientific} & \raisebox{-5pt}{PageRank algorithm execution with igraph.} \\
        \hline
        dna-visualisation & \raisebox{-5pt}{Scientific} & Generating visualization data from DNA sequences. \\
        \hline
        dynamic-html & \raisebox{-5pt}{Webapps} & Rendering dynamic HTML content from a template. \\
        \hline
        \raisebox{-5pt}{uploader} & \raisebox{-5pt}{Webapps} & Uploading files from a given URL to cloud storage. \\
        \hline
        \raisebox{-5pt}{compression} & \raisebox{-5pt}{Utilities} & Compressing multiple files into a ZIP archive for download. \\
        \hline
        image-recognition & \raisebox{-5pt}{Inference} & Performing image classification using ResNet and PyTorch. \\
        \hline
        thumbnailer & Multimedia & Creating thumbnail images from input pictures. \\
        \hline
        video-processing & \raisebox{-5pt}{Multimedia} & Adding watermarks and generates GIFs from video files. \\
        \bottomrule
    \end{tabularx}
\end{table}

\parabf{Workload generation.} We generate simulated user traffic based on the 
Azure Functions Dataset~\cite{ShahradFGCBCLTR20}. For each benchmark, a 
corresponding function invocation trace is randomly sampled from the top 20\% 
most frequently invoked functions—a subset reported to account for the majority 
of resource consumption in production environments~\cite{ShahradFGCBCLTR20}, 
thus representing the critical bottleneck for cluster performance optimization. 
The request rate of each trace is proportionally scaled so that the aggregated 
resource demand approaches the total capacity of the cluster, while the invocation 
frequencies across the ten benchmarks are adjusted to be nearly identical (with a 
relative difference of less than 2\%). Since the original trace data is recorded 
at minute-level granularity, we model intra-minute request arrivals using a 
Poisson process. In alignment with common industrial practice~\cite{aws_scaling_window, 
knative_scaling_window}, the scaling interval is set to 1 minute.

To simulate the real-world characteristic where the resource demand of a serverless 
function follows a normal distribution across multiple executions, we first 
randomly select ten functions from the dataset and compute the variance of their 
resource usage after Gaussian fitting. We then randomly adjust the input data 
size and parameters of each benchmark so that the actual resource requirements 
align with the sampled variance. Given that the dataset only reflects the distribution 
of memory resource demands, we assume a fixed ratio between CPU and memory requirements 
for each benchmark. Accordingly, CPU allocation is determined proportionally 
based on the memory~demand.

\parabf{Cluster configuration.} Our experiments are conducted on CloudLab using 
a cluster consisting of 9 c6620 instances (56 cores, 2 * Intel Xeon Gold 5512U 
@ 2.1GHz, 128GB memory). Eight of them serve as working servers and one as the 
scheduler and user request generator. To simulate the varied resource availability 
conditions in production clusters, our evaluation allocates only a subset of total 
resources on each worker server. Unless otherwise specified, the available resource 
proportions across different servers are sampled from a normal distribution with 
a standard deviation of 0.85.

\parabf{Baseline.} We evaluate \sysname against two state-of-the-art resource 
management approaches for serverless platforms: Palette~\cite{abdi2023palette} 
and FaaSCache~\cite{ASPLOS21:FaaSCache}, which optimize cluster efficiency from 
complementary perspectives. \textbf{Palette} enhances spatial locality by allowing 
users to assign color labels to functions and co-locating those with the same color 
on the same server. In our experiments, each benchmark is assigned a unique color. 
Inspired by caching systems, \textbf{FaaSCache} employs a Greedy-Dual keep-alive 
strategy that assigns a priority to each kept-alive instance and evicts low-priority 
instances first under resource pressure. To ensure a fair comparison and minimize 
the impact of engineering artifacts, we reimplement both baselines in our prototype 
environment. Furthermore, to better reflect real-world production settings while 
maintaining controlled conditions, we enable Kubernetes vertical pod autoscaling 
(VPA) across all systems during evaluation (see prototype implementation details 
in Section~\ref{sec:implementation}).

\parabf{Metric.} We evaluate system performance using memory utilization, CPU 
usage, and task throughput as primary metrics. To verify that the scheduling 
algorithm in \sysname does not lead to task starvation, we also compare the 95th 
percentile tail latency of task processing. Each data point is collected from 
one-hour-long test runs, with the performance data from the first five minutes 
excluded to eliminate the impact of initial cold starts.

\subsection{Benefits of \sysname}
\label{sec:experiments_comparsion}

\begin{figure*}[ht]
    \centering
    \subfloat[]{\includegraphics[width=0.32\linewidth]{figures/Eval/distribution_CPU Utilization.pdf}%
    \label{fig:eval_dis_cpu}}
    \hfil
    \subfloat[]{\includegraphics[width=0.32\linewidth]{figures/Eval/16worker_distribution_Memory Utilization.pdf}%
    \label{fig:eval_dis_memory}}
    \hfil
    \subfloat[]{\includegraphics[width=0.32\linewidth]{figures/Eval/distribution_Task throughput.pdf}%
    \label{fig:eval_dis_throughput}}
    \caption{Performance comparison under different cluster resource distribution. (a) CPU Utilization. (b) Memory Utilization. (c) Task throughput.}
    \vspace{-15pt}
\label{fig:eval_dis}
\end{figure*}

To better demonstrate the performance of \sysname under different deployment 
scenarios, we first compare the overall performance among \sysname and baselines 
under different cluster resource distribution, benchmark number and workload distributions.

\parabf{Comparison under different cluster resource distribution.} Figure~\ref{fig:eval_dis} 
compares the performance of \sysname and baselines under different cluster resource 
distribution. We introduce two resource availability patterns: uniform 
and Zipf distributions. Under the uniform distribution, each server is configured 
with 75\% of its total resources available. In the Zipf-based distribution, the 
fraction of available resources per server follows a Zipf distribution parameterized 
with $\alpha = 0.8$ across $n = 8$ resource tiers. 

Across all resource distribution patterns, \sysname consistently improves both 
CPU and memory utilization, achieving relative improvements of 43.61\%$\sim$55.53\% 
and 41.41\%$\sim$67.01\% over the baselines, respectively. Furthermore, \sysname 
demonstrates performance gains in task throughput, with improvements ranging from 
51.52\% to 103.71\%.

To address the potential concern of task starvation, we examine the impact on 
request-level performance. As shown in Table~\ref{tab:latency}, \sysname reduces 
the overall 95th percentile latency by 30.63\%$\sim$42.42\% compared to the baselines. 
This result indicates that the efficiency gains of \sysname are not achieved by 
systematically delaying particular requests or task types. Instead, the improvement 
in resource utilization and throughput is achieved alongside enhanced and more 
predictable end-to-end performance, even when co-locating diverse benchmarks with 
varying resource demands and execution characteristics.

The performance gains of \sysname are primarily attributed to two mechanisms. 
First, the workload-splitting-based instance placement strategy enables individual 
servers to serve fewer functions, thereby increasing instance reuse probability. 
This approach also aligns server workload with resource capacity, avoiding resource 
waste caused by load-resource mismatch. Second, the vertical-scaling-based task 
scheduling policy reduces both resource over-provisioning and cold-start frequency, 
further boosting overall resource utilization.

\begin{table}[ht]
    \vspace{-10pt}
    \caption{Performance comparison under multiple workflows.\label{tab:latency}}
    \begin{tabularx}{0.47\textwidth}{ 
        >{\raggedright\arraybackslash}X 
        >{\centering\arraybackslash}X
        >{\centering\arraybackslash}X
        >{\centering\arraybackslash}X }
        \toprule
        \multirow{2}{*}{\makecell[l]{\textbf{Resource} \\ \textbf{Distribution}}} & \multicolumn{3}{c}{\textbf{95th Tail Latency (s)}} \\ \cline{2-4}
         & \raisebox{-1pt}{\textbf{Palette}} & \raisebox{-1pt}{\textbf{FaaSCache}} & \raisebox{-1pt}{\textbf{\sysname}} \\
        \hline 
        Norm-85 & 1071 & 1104 & \textbf{743} \\
        Uniform-75\%& 1054 & 1063 & \textbf{729} \\
        Zipf-80 & 1372 & 1335 & \textbf{790} \\
        \bottomrule
    \end{tabularx}
\end{table}

\begin{figure*}[ht]
    \centering
    \subfloat[]{\includegraphics[width=0.32\linewidth]{figures/Eval/func_number_CPU Utilization.pdf}%
    \label{fig:eval_function_cpu}}
    \hfil
    \subfloat[]{\includegraphics[width=0.32\linewidth]{figures/Eval/func_number_Memory Utilization.pdf}%
    \label{fig:eval_function_memory}}
    \hfil
    \subfloat[]{\includegraphics[width=0.32\linewidth]{figures/Eval/func_number_Task throughput.pdf}%
    \label{fig:eval_function_throughput}}
    \caption{Performance comparison under different function numbers. (a) CPU Utilization. (b) Memory Utilization. (c) Task throughput.}
    \vspace{-15pt}
\label{fig:eval_function}
\end{figure*}

\parabf{Comparison under different benchmark number.}
Next, we evaluate \sysname and the baselines under different workload scales. 
Beyond the default configuration of 10 benchmarks, we provide two additional scenarios:

\begin{itemize}
    \item \textbf{5 benchmarks}: A randomly selected subset comprising \textbf{video-processing}, 
    \textbf{graph-pagerank}, \textbf{image-recognition}, \textbf{dna-visualization}, 
    and \textbf{thumbnailer}. To maintain the comparable total load, the 
    invocation rate for each function in this set is doubled;

    \item \textbf{20 benchmarks}: The original 10 benchmarks plus their replicas. Each 
    replica has the same code, resource profile, and runtime as the original but 
    is considered a distinct function, preventing instance reuse between the 
    original-replica pairs. The invocation rate per function is halved to 
    maintain the aggregate request rate.
\end{itemize}

As shown in Figure~\ref{fig:eval_function}, the performance advantage of \sysname 
over the baselines grows as the number of benchmarks increases from 5 to 20. 
Specifically, the improvement in CPU utilization rises from 17.40\%$\sim$16.90\% 
to 44.89\%$\sim$44.06\%, memory utilization gains increase from 41.81\%$\sim$43.26\% 
to 55.42\%$\sim$57.37\%, and task throughput improvements expand from 45.22\%$\sim$51.06\% 
to 75.45\%$\sim$78.98\%.

This scaling trend can be attributed to the workload-splitting strategy. With 
fewer benchmarks, each server already operates on a limited set of functions, 
leaving limited room for further optimization through instance reuse. In contrast, 
when the workload is larger and more diverse, the strategy reduces the number of 
distinct functions each server handles. This effect elevates the instance reuse 
rate, which in turn improves overall~efficiency.

\begin{figure*}[ht]
    \centering
    \subfloat[]{\includegraphics[width=0.32\linewidth]{figures/Eval/skew_CPU Utilization.pdf}%
    \label{fig:eval_workload_cpu}}
    \hfil
    \subfloat[]{\includegraphics[width=0.32\linewidth]{figures/Eval/skew_Memory Utilization.pdf}%
    \label{fig:eval_workload_memory}}
    \hfil
    \subfloat[]{\includegraphics[width=0.32\linewidth]{figures/Eval/skew_Task throughput.pdf}%
    \label{fig:eval_workload_throughput}}
    \caption{Performance comparison under different workload distributions. (a) CPU Utilization. (b) Memory Utilization. (c) Task throughput.}
    \vspace{-15pt}
\label{fig:eval_workload}
\end{figure*}

\parabf{Comparison under different workload distributions.} 
Finally, we examine the impact of workload distribution on scheduling performance. 
In addition to the default resource distribution (see~\ref{sec:experiments_setup}), 
we introduce a Zipf-based load pattern. Specifically, we adjust the $\alpha$ parameter 
of a Zipf distribution (with n = 10) to obtain a probability set that approximates 
the Pareto principle (\emph{i.e.}, the 80/20 rule), where the sum of the two largest 
probabilities is about 0.8. We then scale the request rate of each benchmark 
according to this distribution, preserving the total aggregate invocation frequency 
while ensuring that the two most frequently invoked functions—randomly chosen as 
\textbf{thumbnailer} and \textbf{video-processing}—collectively account for 80\% 
of all invocations.

It can be observed from Figure~\ref{fig:eval_workload} that, compared to the default 
uniform distribution, the optimization margin of \sysname under the Zipf workload 
distribution is relatively smaller. Under the Zipf distribution, \sysname achieves 
an increase in CPU utilization of 23.20\%$\sim$23.58\%, memory utilization of 
27.04\%$\sim$28.50\%, and task throughput of 29.20\%$\sim$31.34\% over the baselines.

\looseness=-1
This result stems from our experimental configuration: the Zipf distribution 
adhering to the 80/20 rule implies that only two functions bear the majority of 
the workload, which naturally grants them more instance reuse opportunities. In 
contrast, while the real-world workload distribution also exhibits skew, it 
involves a larger number of high-load functions competing for server resources 
simultaneously. This competition creates more optimization potential for the 
\sysname strategy. Moreover, even with only two high-load functions, \sysname 
still achieves around 20\%$\sim$30\% higher resource utilization and task throughput 
than the baselines. This gain is primarily because the vertical-scaling-based 
task scheduling policy reduces both instance resource over-provisioning and 
cold-start frequency through task reordering, thereby enhancing overall 
resource~efficiency.

\begin{figure}[t]
    \centering
    \subfloat[]{\includegraphics[width=0.9\linewidth]{figures/Eval/detail_cpu.pdf}%
    \label{fig:eval_detail_cpu}}
    \hfil
    \subfloat[]{\includegraphics[width=0.9\linewidth]{figures/Eval/detail_memory.pdf}%
    \label{fig:eval_detail_memory}}
    \caption{Detail Resource Comsupmtion of \sysname and baselines. (a) CPU Detail Comsupmtion. (b) Memory Detail Comsupmtion.}
    \vspace{-15pt}
\label{fig:eval_detail}
\end{figure}

\parabf{Performance analysis.} To further demonstrate the reasons for \sysname 
performance advantage, Figure~\ref{fig:eval_detail} provides a detailed 
illustration of the average resource consumption of \sysname and the baselines 
during task processing. We categorize CPU and memory resources into the following 
four types: 

\begin{itemize}
    \item \textbf{Task Execution}: resources allocated for processing tasks;
    \item \textbf{Cold Start}: resources consumed during the initialization of instances;
    \item \textbf{Standby Instance}: resources held by ready but currently idle instances;
    \item \textbf{Idle Resource}: available but unallocated resources.
\end{itemize}

Taking CPU utilization as an example, compared to the baselines, \sysname increases 
the resource share for Task Execution by 45.47\%$\sim$43.61\%, for Standby Instance 
by 110.87\%$\sim$125.58\%, and for Idle Resource by 43.59\%$\sim$47.37\%, while 
it reduces the share for Cold Start by 83.65\%$\sim$83.56\%. A similar trend is 
observed for memory utilization.

The resource utilization advantage of \sysname stems from the reduction in cold 
starts. This is achieved through two strategies. First, the workload-splitting-based 
instance placement concentrates a narrower set of functions on each server, which 
increases instance reuse opportunities. Second, the vertical-scaling-based task 
scheduling policy reorders tasks and dynamically adjusts instance resources, 
enabling instances to serve varying demands throughout their lifecycle without 
repeated cold starts or static over-provisioning.

The resources conserved from fewer cold starts are reallocated in two ways: they 
increase the share for task execution while expanding the pool of standby instances. 
Consequently, a larger ready-to-use instance pool is maintained, raising the 
probability of warm starts.

\subsection{Effectiveness of \sysname}
\label{sec:experiments_technique}

{
\begin{figure}[t]
    \centering
    \includegraphics[width=0.97\linewidth]{figures/Eval/workflow_xiaorong.pdf}
    \caption{Performance comparison between \sysname and simplified versions.}
    \label{fig:technique}
    \vspace{-10pt}
\end{figure}
}

In this section, we investigate the impact of the techniques employed by 
\sysname. We measure \sysname with the following two simplified versions:
\begin{itemize}
    \item \textbf{Placement*}: This variant uses only the workload-splitting-based 
    instance placement strategy, while tasks within each server are scheduled using 
    a simple FIFO policy.
    \item \textbf{Scheduling*}: This variant uses only the vertical-scaling-based 
    task scheduling policy within each server, while user requests are randomly 
    assigned across servers.
\end{itemize}

Figure~\ref{fig:technique} shows the performance comparison between \sysname and 
simplified versions. We also provide the evaluation results of the baselines for 
reference.

As shown, the performance of the Placement* version lies between the baselines 
and \sysname. Compared to the baselines, it achieves improvements of 20.07\%$\sim$21.63\%, 
16.90\%$\sim$17.23\%, and 17.31\%$\sim$15.74\% in CPU utilization, memory utilization, 
and task throughput, respectively. However, it still falls short of \sysname by 
16.39\%, 17.33\%, and 23.62\% in the metrics. These results demonstrate that the 
workload-splitting-based instance placement strategy alone can effectively enhance 
scheduling performance through load partitioning. Furthermore, it requires integration 
with the vertical-scaling-based task scheduling policy for fine-grained, 
server-internal resource management to achieve optimal performance.

The performance of the Scheduling* variant reveals a more nuanced trade-off. It 
underperforms \sysname in resource utilization, with CPU and memory utilization 
lower by 13.35\% and 29.99\%, respectively. Despite this, it achieves a 27.95\% 
higher task throughput. This discrepancy occurs because Scheduling* prioritizes 
tasks with lower resource demands, shorter execution times, and higher invocation 
frequencies, without explicitly optimizing for overall server utilization.

When integrated with the workload-splitting-based instance placement strategy 
(as in \sysname), each server handles a limited, resource-complementary set of 
functions. This increases the likelihood that tasks with high resource demands 
are selected for instance setup and execution, raising the execution priority for 
subsequent tasks from the same function (see Section~\ref{sec:scheduling}). In 
contrast, without effective load partitioning, a wider variety of instances compete 
for server resources. In this scenario, resource-intensive instances are served 
mainly through starvation-avoidance mechanisms, receiving only limited execution 
opportunities. The resulting contention leads to frequent instance eviction and 
cold starts, which hinders effective improvement in resource utilization.

\subsection{Scalability}
\label{sec:experiments_overhead}

Finally, we evaluate the scalability of \sysname. Due to testbed limitations, 
this section includes both real and simulated experiments. We first demonstrate 
the performance of \sysname on a larger-scale cluster. Then we
show the system overhead of the \sysname scheduler on up to 1024 nodes.

\begin{figure*}[ht]
    \centering
    \subfloat[]{\includegraphics[width=0.32\linewidth]{figures/Eval/16worker_distribution_CPU Utilization.pdf}
    \label{fig:eval_big_cpu}}
    \hfil
    \subfloat[]{\includegraphics[width=0.32\linewidth]{figures/Eval/16worker_distribution_Memory Utilization.pdf}
    \label{fig:eval_big_memory}}
    \hfil
    \subfloat[]{\includegraphics[width=0.32\linewidth]{figures/Eval/16worker_distribution_Task throughput.pdf}
    \label{fig:eval_big_throughput}}
    \caption{Performance comparison in 16-worker cluster. (a) CPU Utilization. (b) Memory Utilization. (c) Task throughput.}
    \vspace{-15pt}
\label{fig:eval_big}
\end{figure*}

\parabf{Larger-scale cluster experiment.}
In the larger-scale cluster experiment, we increased the number of worker nodes 
to 16 and proportionally scaled the request invocation frequency, while keeping 
other conditions constant. Figure~\ref{fig:eval_big} shows the performance comparison 
among \sysname and the baselines. Compared to the baselines, \sysname improves 
CPU utilization by 31.80\%$\sim$61.75\%, memory utilization by 41.95\%$\sim$53.82\%, 
and task throughput by 40.24\%--110.75\%. These results align with those observed 
in the smaller cluster, demonstrating that \sysname maintains its performance 
advantage stably at scale.

{
\begin{figure}[t]
    \centering
    \includegraphics[width=0.97\linewidth]{figures/Eval/overhead_placement.pdf}
    \caption{The average latency of \sysname in handling scaling request.}
    \label{fig:overhead_placement}
    \vspace{-10pt}
\end{figure}
}

{
\begin{figure}[t]
    \centering
    \includegraphics[width=0.97\linewidth]{figures/Eval/overhead_scheduling.pdf}
    \caption{The average latency of \sysname in handling scaling request.}
    \label{fig:overhead_scheduling}
    \vspace{-10pt}
\end{figure}
}

\parabf{System overhead.}
We next evaluate the overhead of the \sysname scheduler under varying cluster 
sizes and request frequencies. To do so, we simulate scheduler load for clusters 
ranging from 64 to 1024 worker servers. To avoid interference from bursty traffic 
patterns, we generate invocations at a steady rate instead of using the trace 
from the Azure Functions Dataset~\cite{ShahradFGCBCLTR20}. The experimental results 
are presented in Figures~\ref{fig:overhead_placement} and~\ref{fig:overhead_scheduling}. 

Figure~\ref{fig:overhead_placement} shows the average latency for generating cluster-wide 
instance placement strategies under different cluster sizes. It can be observed 
that the placement-generation overhead increases gradually with cluster scale but 
remains largely unaffected by request frequency. This is because \sysname scales 
the overall load proportionally to the cluster size before computing the placement 
strategy. Even with 1024 worker nodes and an invocation rate of $250 \times 10^3$ 
requests per minute, the latency stays below 110 ms. This result indicates that 
the scheduler can generate placement strategies promptly, even in large-scale 
clusters, demonstrating the practical scalability of \sysname.

Figure~\ref{fig:overhead_scheduling} shows the scheduling latency for individual 
task assignments under different invocation frequencies. It should be noted that 
placement strategy generation is performed by a separate background process within 
the scheduler. The main scheduling process only receives the resulting placement 
plan and assigns tasks accordingly. Therefore, only the task assignment overhead 
contributes to the additional latency introduced by \sysname in the end-to-end 
task execution path. As shown, even under a heavy load of 1024 worker nodes, the 
extra scheduling delay added by \sysname remains below 10 ms. This overhead is 
negligible for the vast majority of FaaS scenarios, confirming that the scheduling 
logic of \sysname is lightweight enough for production-scale deployment.
