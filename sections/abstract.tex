\begin{abstract}

Serverless computing is an increasingly popular paradigm for cloud applications, offering scalability and pay-per-use billing. However, serverless platforms must efficiently manage resources for a multitude of diverse functions and requests with different inputs. Conventional approaches often ignore the resource requirement differences between functions and individual requests, leading to resource fragmentation, poor cluster utilization, and high overhead from instance initializations.

This paper introduces COSMOS, a novel serverless computing system that improves efficiency by exploiting the complementary resource usage patterns across different functions. Our approach is founded on two key optimizations. First, we co-locate functions with complementary resource requirements to maximize node utilization, while simultaneously grouping instances of the same function to improve locality and reduce associated overhead. Second, we leverage the emerging capability to resize an instance's resource allocation, allowing our system to reuse existing warm instances for requests with heterogeneous resource demands and minimize instance initializations. Experiments on real-world workloads show our system improves resource utilization by up to 55.53\% (CPU) and 67.01\% (memory), and increases overall cluster throughput by up to 2.03$\times$ compared to state-of-the-art serverless computing systems.

\end{abstract}