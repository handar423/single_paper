\begin{abstract}

Serverless computing is an increasingly popular paradigm built on top of data center network (DCN), offering scalability and fine-grained billing. However, serverless platforms must efficiently manage resources for a multitude of diverse functions and requests with different inputs. Conventional approaches often ignore the resource requirement differences between functions and individual requests, leading to resource fragmentation, poor cluster utilization, and high overhead from instance initializations.

This paper introduces COSMOS, a novel serverless computing system that improves efficiency across nodes in a cluster by exploiting the complementary resource usage patterns of different functions. Our approach is founded on two key optimizations. First, we route functions with complementary resource requirements to the same node to maximize node utilization, while simultaneously grouping instances of the same function to improve locality and reduce associated overhead. Second, we leverage the emerging capability to resize an instance's resource allocation, allowing our system to reuse existing warm instances for requests with heterogeneous resource demands and minimize instance initializations. Experiments on real-world workloads show our system improves utilization of multiple resources by 41.41\%--67.01\%, and increases overall throughput by up to 2.03$\times$ compared to state-of-the-art serverless computing systems.

\end{abstract}