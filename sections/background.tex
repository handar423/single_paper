\section{Background and Motivation}
\label{sec:background}

\subsection{Resource Management of Serverless Computing}
\label{sec:background:resource-management}

% 1. serverless的资源管理模式: 静态资源分配管理
% 把 resource allocation 分为两个步骤,一个是节点-level的资源分配,即运行时的核数和内存,另一个cluster-level的资源分配,即是函数容器副本扩缩容和放置,即决定副本数量、并将容器放置到具体的节点。

Serverless computing has emerged as a promising cloud paradigm, allowing developers to focus on application logic by abstracting away infrastructure tasks such as provisioning, scaling, and maintenance. In this event-driven model, users simply provide code (functions) with basic resource specifications and triggers. The platform then orchestrates the entire lifecycle\textemdash provisioning sandboxed environments, performing on-demand autoscaling, and managing pay-per-use billing. Consequently, efficient resource management within these platforms is essential to ensure high performance, maximize utilization, and minimize costs.

In practice, resource management in serverless platforms encompasses two primary dimensions: i) per-instance resource configuration, which defines the resource limits for individual function sandboxes; and ii) per-function orchestration, which governs autoscaling (instance count) and placement (node selection). Currently, both dimensions rely on static and uniform settings, as detailed below. \junyi{This is a very strong argument. I doubt this. should ask Bowen to double check}

\parabf{Per-instance resource allocation.}
% 1) 静态的运行时资源配置:介绍 lambda 支持指定固定资源规格,及其对延迟和开销的影响,所以选取一个所有请求都不违背qos的配置
Existing serverless computing systems from both industry~\cite{Doc:Azure_Functions_Hosting,
Doc:GCP_CloudRun_Mem, aws_lambda} and academia~\cite{ASPLOS21:FaasCache, shahrad2020serverless, wang2018peeking, pu2019shuffling}
allocate a uniform amount of resources for all instances of a given function. In AWS Lambda, a developer specifies a memory size (e.g., 128 MB--10,240 MB), from which the platform derives proportional vCPU and network bandwidth shares~\cite{aws_lambda}. Developers may also set ephemeral storage per instance (default 512 MB, configurable up to a few gigabytes). These limits remain fixed throughout the sandbox's lifetime. Similar static per-instance specifications are used by Azure Functions~\cite{Doc:Azure_Functions_Hosting} and Google Cloud Run~\cite{Doc:GCP_CloudRun_Mem}. This approach presumes uniform resource demand across function invocations. A user typically selects the smallest configuration that satisfies the function's QoS (e.g., latency SLO) to minimize costs. \todo{add some citations}

%  已有工作: 运行时资源配置:根据 profile 进行 static 资源分配,在延迟和开销间做合理tradeoff,引用一些相关工作,如 
Optimizations on the static per-instance resource allocation focus on
profile-guided parameter tuning~\cite{aws_lambda_power_tuning,ASPLOS21:FaasCache, shahrad2020serverless, wang2018peeking, pu2019shuffling} \todo{jolteon and ditto?}
By profiling functions across discrete resource configurations, this method derives latency–cost curves to identify the cheapest configuration satisfying specific SLOs. An automated tool can then use these profiles to minimize costs under QoS constraints by exploiting per-function heterogeneity.

\parabf{Instance scaling and placement.}
% 2)根据静态资源配置进行容器扩缩容放置:现有的 serverless 系统大多基于静态资源配置进行容器放置调度
% 仅容器启动、离开时比较资源:只需维护节点已分配资源和未分配资源总计,任务到来时可以分配到未分配资源>静态请求资源的节点上,更新已分配资源、未分配资源,函数容器销毁时释放资源、更新已分配资源、未分配资源即可
Existing serverless systems perform scaling~\cite{aws_scaling_window, aws_scaling_eq, knative_scaling} and placement based on a static per-function resource requirement. While autoscalers adjust replica counts in response to load, schedulers employ a simple resource accounting model where each node tracks allocated versus available capacity. A container is admitted to a node only if the available capacity meets the function’s static requirement (e.g., memory, vCPU). Crucially, these resource counters are updated only during lifecycle events\textemdash decremented upon placement and incremented upon termination\textemdash ignoring actual usage during execution.

% 已有工作:

% 第一段,函数容器副本数量:基于历史预测,引用文章
For autoscaling, recent systems research uses prediction to
right-size replicas and proactively prewarm functions. Shahrad et al.
analyze production traces and propose a per-function histogram policy
that jointly sets keep-alive and pre-warming windows, cutting cold
starts at lower cost~\cite{shahrad2020serverless}. For workflows,
Orion models end-to-end SLOs in DAGs and prewarms downstream stages
with the right look-ahead~\cite{ORION_MahgoubYSECB22}.

% 第二段(或二、三段),函数容器副本放置缓存亲和性、多资源综合考虑调度:考虑(先几句话说缓存为啥重要)缓存是否存在、多资源性质(即预备资源分配),引用文章
% 可以考虑,包含 FaasCache这种单机容器缓存管理, serverless workflow 相邻stage亲和放置,jiagu这种多节点协同缓存管理,引用更多文章
For placement, Recent advancements consider three main aspects: cache awareness,
heterogeneous resource, and data-affinity, improving cost-efficiency and latency in serverless systems.
Cache-aware strategies reduce cold start overheads by enabling instances to share initialized memory states~\cite{SOCK_OakesYZHHAA18, li2022help}, allowing for rapid and cost-effective instantiation. Heterogeneity-aware placement~\cite{RoyPT22, DuLJXZC22} optimizes cluster utilization by matching diverse function resource profiles to specific hardware capabilities, thereby minimizing waste. Finally, data-affinity strategies~\cite{Wukong_CarverZWAWC20, ORION_MahgoubYSECB22} focus on serverless workflows, co-locating dependent tasks to reduce data transfer overheads between stages.
%Incorporating cache awareness helps to decrease costs and latency associated with cold starts.
%Different instances can share certain memory states~\cite{SOCK_OakesYZHHAA18, li2022help},
%allowing for rapid and cost-efficient instance creation.
%Moreover, considering heterogeneous resource awareness~\cite{RoyPT22, DuLJXZC22} in placement is critical,
%as functions and resources exhibit heterogeneous characteristics.
%This approach minimizes resource wastage and optimizes utilization.
%Additionally, data-affinity aware placement strategies~\cite{Wukong_CarverZWAWC20, ORION_MahgoubYSECB22}
%have been proposed for serverless workflows to reduce data transfer overheads between stages.




% 2. serverless 函数资源需求分析
\subsection{Analysis of Resource Requirements of Functions}
\label{sec:background:demand-analysis}

\begin{figure}[t]
  \centering
  \includegraphics[width=0.95\linewidth]{figures/background/cpu_memory_ratio_histogram.pdf}
  \caption{Heterogeneous CPU:memory ratios across functions.}
  \label{fig:cpu_mem_ratio_hist}
\end{figure}

% 1) 函数级别需求分布
\paraf{Resource Usage Variation Across Functions.}
The resource demands of serverless functions are highly heterogeneous. We quantify this heterogeneity by analyzing the execution traces of 162 real-world functions from the open-source Huawei dataset~\cite{huawei_dataset}, calculating the average CPU-to-memory ratio (core/GB) for each. Figure~\ref{fig:cpu_mem_ratio_hist} presents the distribution of these ratios across all functions.\footnote{The statistical results shown are from [Request Table Module, Region 4] in dataset~\cite{huawei_dataset}. We observed similar distributions in other parts of the dataset. \todo{what does this mean? confusing}} The distribution is highly dispersed, with ratios ranging from 0.09 to 9.20 and a high CV of 62.43\%, indicating that some functions are intensely compute-intensive while others are predominantly memory-intensive. This substantial variation in intrinsic resource profiles implies that a scheduler unaware of such differences is prone to creating resource fragmentation, where one resource type is exhausted on a node while another remains underutilized.

\begin{figure}[t]
    \centering
    \includegraphics[width=0.95\linewidth]{figures/background/cdf_cpu_cv_regions.pdf}
    \caption{Resource usage variations of serverless function.}
    \vspace{-15pt}
\label{fig:usage_variation}
\end{figure}

\begin{figure}[t]
  \centering
  \includegraphics[width=0.95\linewidth]{figures/background/cpu_density_4funcs_134_127_170_206.pdf}
  \caption{Representative distributions of function resource consumption.}
    \vspace{-15pt}
  \label{fig:usage_patterns}
\end{figure}

% 2) 调用级别资源占用
\paraf{Resource Usage Variation Across Invocations.}
Resource demand also exhibits significant variation across different invocations of the same function. We compute  the CV for CPU usage of each function across its invocations. The distribution of these CV values is presented in Figure~\ref{fig:usage_variation}. Our analysis reveals that for a substantial portion (54\%) of functions, the CV exceeds 50\%, indicating that these functions have a highly dispersed resource demand profile across invocations. A further 26\% of functions exhibit a moderate degree of variation, with CV values between 15\% and 50\%. Only 20\% of functions demonstrate relatively stable resource requirements (CV < 15\%). \junyi{is this classification standard?}

To provide deeper insight into these statistical patterns, Figure~\ref{fig:usage_patterns} 
illustrates four representative resource consumption distributions observed in 
the trace: (a) a quasi-normal distribution (where the truncated right tail is 
attributed to the imposed per-instance resource limit); (b) a long-tailed 
distribution; \junyi{does not look like a standard long-tail distribution. can we have a standard one?} (c) a concentrated distribution; and (d) certain atypical cases 
that do not conform to standard distributions, potentially influenced by specific 
input parameter distributions or external factors. This intra-function variability 
necessitates a scheduling mechanism that can adapt to dynamic, request-level 
resource needs rather than relying on a static capacity. \junyi{this observation is interesting. might want to do more analysis and explain why.}


% % 2) 现有调度方案的问题
% \parabf{Drawback of existing solutions: Static resource allocation waste resources.}
% Static, uniform per‑instance sizing ignores intra‑function resource usage variability.
% When set to a high tier, most invocations are overprovisioned, wasting CPU/memory
% and inflating cost. When set to a low tier, peak‑demand invocations are
% underprovisioned, causing throttling, queuing, memory overflow.
% Fixed sizing also worsens data locality and resource fragmentation.

% 3) Opportunity: 函数动态运行时资源调整 
% 现有容器框架支持动态运行时资源分配(具体机制或者接口),以及其好处(节省资源减少浪费)
\parabf{Opportunity: Dynamic in‑place runtime resource adjustment}
A key opportunity to address the intra-function variability lies in recent container 
runtime enhancements that support dynamic, in-place resource adjustment. Specifically, 
platforms like Kubernetes now allow the resource allocations (CPU and memory) of 
a running pod to be updated without restarting it, via mechanisms such as the Pod 
resize subresource and live cgroup updates~\cite{Doc:K8s_Pod_Resize, Doc:K8s_CRI_Update}.
This capability enables a container's assigned resources to be aligned in real 
time with the actual demand of the request it is processing. It thus provides a 
foundational mechanism to move beyond static per-function allocation, toward a 
fine-grained model where resources can be expanded or contracted per invocation. 
It absorbs demand bursts, reduces stranded resources from over-provisioning, 
and lowers the risk of SLO violation\textemdash while preserving performance 
benefits of a warm sandbox.

% Kubernetes enables in‑place vertical scaling without pod recreation via the Pod 
% resize subresource, allowing live updates to both CPU and memory resource requests 
% and limits. CPU resources can be adjusted by altering cgroup CPU shares and quotas, 
% allowing the kubelet to allocate CPU cycles according to usage metrics and policy 
% constraints. Similarly, memory scaling is achieved by modifying cgroup limits,
% which dynamically changes the memory allocation for a running container without 
% stopping it.\cite{Doc:K8s_Pod_Resize, Doc:K8s_CRI_Update}. Dynamically aligning 
% each invocation's resource demands with its sandbox allocation absorbs bursts and 
% handles intra-function heterogeneity, improving utilization and reducing 
% overprovisioning and SLO risks. This ensures that all resources can be increased 
% or decreased to respond to real-time demands. A serverless control plane can 
% utilize these features to dynamically adjust sandboxes for both CPU and memory, 
% optimizing resources while preserving state and improving cost and latency efficiency.

% 3. 动态资源分配的挑战

\subsection{Challenges in Dynamic Resource Allocation for Functions}
\label{sec:background:challenges}

% 1)挑战1:副本放置挑战:动态资源分配打破了原有的静态资源分配假设,容器的资源需求不再是固定的,不能仅在容器启动和销毁时考虑节点资源分配
% strawman solution:发生扩容剩余节点资源不足时直接evict其他同节点容器,但带来缓存损失,且可能放大(相对较大规格的容器扩容,可能一下需要evict多个小规格容器)
% appeal for Minimal-Disruption Placement Scaling with multi-resource balance awareness

\paraf{Challenge 1: Placement under dynamic in-place resizing.} Dynamic in-place resizing disrupts the fixed-size bin-packing model: a sandbox’s footprint may expand while active, making node feasibility assessments at start/stop times inadequate. A simplistic approach is to evict co-located sandboxes to make space when upscaling faces insufficient capacity. However, this naive strategy risks evicting other co-located function instances with increased resource demand. This issue is exacerbated when a large upsizing occurs under constrained resource availability, resulting in eviction amplification: a single significant resize event can force the eviction of multiple smaller co-located sandboxes, thereby hurting cache locality and increasing tail latency. Additionally, fragmentation can occur when resource demand descreases. These problems may lead to low resource utilization and application throughput.

Therefore, We require an intelligent placement strategy that is aware of multi-resource needs \junyi{should be aware of the need of resizing?}, minimizing sandbox disruption. It should balance the CPU and memory usage per node to prevent fragmentation, employing targeted part-migrations \junyi{what is this, confusing} only when they minimize overall disruption. Such policies encourage managed churn, safeguard warm states, and enhance packing efficiency and throughput during dynamic resizing.

% todo: 请按需加图

% \begin{figure}[t]
%   \centering
%   \begin{minipage}[t]{0.48\linewidth}
%     \centering
%     \includegraphics[width=\linewidth]
%     {figures/background/challenge2_eviction_amplification.png}
%     \vspace{2pt}
%     \small Left: Upsizing a large sandbox may evict many small ones
%     (amplification).
%   \end{minipage}\hfill
%   \begin{minipage}[t]{0.48\linewidth}
%     \centering
%     \includegraphics[width=\linewidth]
%     {figures/background/challenge2_stranding.png}
%     \vspace{2pt}
%     \small Right: Ignoring multi‑resource balance strands CPU or memory.
%   \end{minipage}
%   \caption{Challenge 2: Dynamic in‑place scaling complicates placement under
%   tight capacity.}
%   \label{fig:challenge2_placement}
% \end{figure}


% 3)挑战3:资源缩容本身有成本挑战:扩容成本低,但缩容成本高。对于请求量没那么大的函数有难度。
%Second, dynamic resource adjustment mechanisms introduce non-negligible overhead.
%Specifically, while increasing a container's resource capacity is a relatively fast operation, decreasing it is often much slower.
%For example, reducing the memory capacity of a container requires reclaiming memory pages, which incurs substantial latency.
%As resource adjustment occurs on the critical path of request execution, this latency directly increases the end-to-end request handling time.
% add overhead detail

% 对于请求量没那么大的函数
% strawman 1: 起很多副本,细致应对不同资源需求,但导致大量浪费
% strawman 2:只起少量副本,一会儿扩一会儿缩,但频繁调整带来高开销
% appeal for(opportunity): Vertical Scaling-Based Task Scheduling Algorithm,即按请求资源量从小到大reorder请求,使得单容器能从小到大单调递增扩容,然后只减少一次到很小,循环往复
\paraf{Challenge 2: High overhead of downsizing.} Dynamic in-place scaling is asymmetric: upsizing is swift, but downsizing incurs high costs. Increasing the memory allocation takes effect immediately, whereas reducing it triggers cgroup reclamation, page-cache eviction, and compaction, potentially stalling both runtime and kernel. Since resizing affects the request path, slow reclaim increases tail latency and queueing. Concurrent shrinking actions can cause reclamation storms \junyi{do we want to introduce new term?}, lowering node throughput even when average capacity suffices.

Two baseline strategies struggle to handle bursty, diverse functions effectively. The first approach is to over-provision numerous replicas, each fixed to a narrow size band, to avoid the need for resizing. However, this leads to significant resource fragmentation and poor utilization due to excessive headroom. The second approach is to maintain fewer replicas and resize them dynamically for every request. Unfortunately, such frequent adjustments incur high memory reclamation overheads, disrupt page caches and JIT states, and increase scheduler interference. Ultimately, both strategies degrade system throughput, negating the theoretical advantages of dynamic scaling.

\junyi{we should talk about what's the requirement instead of what we actually do} We propose a monotonic scale-up scheduler that orders invocations by predicted
demand within short batching windows. Each sandbox handles small, then medium,
then large requests, growing steadily and shrinking at most once per window during
quiescence. This approach amortizes reclamation, maintains warm state, and
alleviates pressure. Figure~\ref{fig:vertical_scaling_scheduling} demonstrates
the benefits of this method over naive resizing with FIFO scheduling. By reordering
requests, the scheduler reduces downsizing overheads by 33.3\%, leading to lower
overall latency and higher system throughput.
\begin{figure}[t]
  \centering
  \subfloat[]{\includegraphics[trim=0 80pt 400pt 0, clip, width=0.47\linewidth]{figures/background/vertical.pdf}}
  \hfil
  \subfloat[]{\includegraphics[trim=400pt 80pt 0 0, clip, width=0.47\linewidth]{figures/background/vertical.pdf}}
  \caption{Vertical scaling-based task scheduling mitigates downsizing overhead
  by reordering requests. (a) Naïve resizing, Cost = 12. (b) Monotonic Scale-up, Cost = 8.}
  \vspace{-15pt}
  \label{fig:vertical_scaling_scheduling}
\end{figure}

% # 本文优化目标(场景:高负载、高并发,核心目标:资源利用率和任务吞吐量)