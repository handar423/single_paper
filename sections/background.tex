\section{Background and Motivation}
\label{sec:background}

\subsection{Existing resource management in Serverless}
\label{sec:background:resource-management}

% 1. serverless的资源管理模式: 静态资源分配管理
% 把 resource allocation 分为两个步骤,一个是节点-level的资源分配,即运行时的核数和内存,另一个cluster-level的资源分配,即是函数容器副本扩缩容和放置,即决定副本数量、并将容器放置到具体的节点。

\paraf{Static runtime resource allocation}
% 1) 静态的运行时资源配置:介绍 lambda 支持指定固定资源规格,及其对延迟和开销的影响,所以选取一个所有请求都不违背qos的配置

\parabf{Static-resource serverless sandbox autoscaling and placement.}
% 2)根据静态资源配置进行容器扩缩容放置:现有的 serverless 系统大多基于静态资源配置进行容器放置调度
% 仅容器启动、离开时比较资源:只需维护节点已分配资源和未分配资源总计,任务到来时可以分配到未分配资源>静态请求资源的节点上,更新已分配资源、未分配资源,函数容器销毁时释放资源、更新已分配资源、未分配资源即可

\parabf{Existing approaches for serverless static resource management.}

% 3) 已有工作:
%   a) 运行时资源配置:根据 profile 进行 static 资源分配,在延迟和开销间做合理tradeoff,引用一些相关工作,如 

Furthermore, existing resource allocation strategies operate at the function level, where each function instance is allocated a static resource capacity~\cite{Doc:AWS_Lambda_Mem, Doc:Azure_Functions_Hosting, Doc:GCP_CloudRun_Mem, ASPLOS21:FaasCache} \todo{add a few more papers with static allocation}. This strategy is predicated on the assumption that a function exhibits uniform resource demands across all its invocations

%   b) 函数容器副本扩缩容和放置
% 第一段,函数容器副本数量:基于历史预测,引用文章
% 第二段(或二、三段),函数容器副本放置缓存亲和性、多资源综合考虑调度:考虑(先几句话说缓存为啥重要)缓存是否存在、多资源性质(即预备资源分配),引用文章
% 可以考虑,包含 FaasCache这种单机容器缓存管理, orion这种单函数缓存副本数控制、跨函数预热,jiagu这种多节点协同缓存管理,引用更多文章

% 2. serverless 函数资源需求分析
\subsection{Function Resource Demand Analysis}
\label{sec:background:demand-analysis}

% 1) 函数资源占用的特点
\paraf{Dynamic resource usage across invocations.}

fig1 different resource usage pattern across different invocations of the same function


% 2) 现有调度方案的问题
\parabf{Drawback of existing solutions: Static resource allocation waste resources.}

static resource allocation for the same function may lead to resource waste or performance degradation



% 3) Opportunity: 函数动态运行时资源调整 
% 现有容器框架支持动态运行时资源分配(具体机制或者接口),其好处


% 3. 动态资源分配的挑战
\subsection{Challenges in Dynamic Resource Allocation for Serverless Functions}
\label{sec:background:challenges}

% 1) 挑战1:资源分配+副本扩缩挑战:如何确定一个函数的容器的资源组合(比如1核1个、2核1个、3核1个),以便在高效利用资源的同时,满足函数的动态资源需求
% appeal for workload-split based 资源组合分配

% 2)挑战2:副本放置挑战:动态资源分配打破了原有的静态资源分配假设,容器的资源需求不再是固定的,不能仅在容器启动和销毁时考虑节点资源分配
% strawman solution:直接evict其他同节点容器,但带来缓存损失,且可能放大(相对较大规格的容器扩容,可能一下需要evict多个小规格容器)
% appeal for Minimal-Disruption Placement Scaling

额外一段,简单说下 fig2 heterogeneous multi-resource requirements characteristics(cpu:memory ratio) across different functions

% 3)挑战3:资源缩容本身有成本挑战:扩容成本低,但缩容成本高。对于请求量没那么大的函数有难度。
%Second, dynamic resource adjustment mechanisms introduce non-negligible overhead.
%Specifically, while increasing a container's resource capacity is a relatively fast operation, decreasing it is often much slower.
%For example, reducing the memory capacity of a container requires reclaiming memory pages, which incurs substantial latency.
%As resource adjustment occurs on the critical path of request execution, this latency directly increases the end-to-end request handling time.
% add overhead detail
对于请求量没那么大的函数
strawman 1: 起很多副本,细致应对不同资源需求,但导致大量浪费
strawman 2:只起少量副本,一会儿扩一会儿缩,但频繁调整带来高开销
appeal for(opportunity): Vertical Scaling-Based Task Scheduling Algorithm,即按请求资源量从小到大reorder请求,使得单容器能从小到大单调递增扩容,然后只减少一次到很小,循环往复

% # 本文优化目标(场景:高负载、高并发,核心目标:资源利用率和任务吞吐量)

add demo experiment here, sota