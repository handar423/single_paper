% \section{Algorithm Design}

\section{\sysname Overview}
\label{sec:overview}

{
\begin{figure}[t]
  \centering
  \includegraphics[width=\linewidth]{figures/overview.pdf}
  \vspace{-15pt}
  \caption{\sysname overview.}
  \vspace{-15pt}
  \label{fig:overview}
\end{figure}
}

Figure~\ref{fig:overview} illustrates the overall architecture of \sysname. The
scheduling framework of \sysname operates at two levels: cluster-wide task 
dispatching and per-server scheduling. At the cluster level, incoming user requests 
are first directed to a task dispatcher, which performs initial traffic accounting. 
The instance placement manager continuously collects traffic statistics, function 
metadata, and task performance metrics, periodically executing the instance 
placement algorithm. The resulting placement plan is fed back to the task dispatcher 
to guide request-to-server assignments. Tasks assigned to worker nodes are placed 
into task queues managed by the worker scheduler, which maintains separate queues 
and priorities for different functions and orchestrates instances within the 
node—executing tasks and performing in-place resource scaling. We introduce two key 
techniques in \sysname: complementary swarming instance placement, and 
monotonic scale-up task scheduling.

\parabf{Complementary Swarming Instance Placement} \sysname controls instance 
placement by establishing a mapping between functions (or, more precisely, subsets 
of requests from the same function that share a given resource-demand range) and 
specific worker servers. This mapping dictates how tasks are assigned across the 
cluster. Based on a heuristic algorithm, \sysname ensures that the collective 
resource demand of the function mix served by each worker node aligns with the 
server resource capacity, thereby maximizing resource utilization. Simultaneously, 
\sysname minimizes the number of distinct functions served by each node to increase 
instance reuse. For functions with high resource demands that require deployment 
across multiple servers, \sysname employs task-level load splitting to distribute 
their workload. By controlling the optimization step size of the heuristic 
algorithm, \sysname guarantees algorithm convergence and keeps the computation 
duration below a hundred milliseconds.

\parabf{Monotonic Scale-up Task Scheduling}
\sysname designs the worker-level scheduling algorithm around the restart-free 
in-place pod scaling feature of Kubernetes. The scheduler maintains dual task 
queues, both of which are sorted in ascending order of resource demand. This 
ordering ensures that, within an appropriate time window, the resource requirements 
of tasks processed by given instances increase monotonically. Consequently, a pod 
can be scaled up incrementally to meet the growing demands of successive 
tasks without over-provisioning, thereby increasing resource utilization without 
incurring restarts. To prevent task starvation, \sysname holistically considers 
task waiting time, cold-start overhead, and resource requirements in designing task 
queue priorities. By strategically reordering tasks to extend queue lengths under 
priority constraints, \sysname effectively increases resource utilization and 
reduces the frequency of cold starts.