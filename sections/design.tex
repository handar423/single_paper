\section{\sysname Design}

\sysname employs a two-level hierarchical approach for scheduling function instances 
and requests. At the inter-server level, complementary swarming placement groups 
and places functions with opposing CPU/memory needs onto the same server, maximizing 
node utilization and reuse. At the intra-server level, monotonic scale-up scheduling 
reorders each function’s requests by increasing resource demand, allowing a single 
sandbox to scale up incrementally and avoid restarts.

\subsection{Complementary Swarming Instance Placement}
\label{sec:design:placement}

The complementary swarming placement algorithm addresses the scheduling challenge 
posed by heterogeneous function resource demands, as quantified in 
\S\ref{sec:background:demand-analysis}. It pursues a dual objective: i) maximizing 
cluster-wide resource utilization by co-locating functions with complementary CPU 
and memory profiles on the same server, thereby mitigating resource stranding; 
and ii) enhancing instance reuse by deliberately narrowing the set of distinct 
functions served per node. To achieve this, the algorithm proceeds iteratively. 
In each round, it assigns the workload of a function to one or more worker nodes 
based on the function‘s resource profile and, when necessary, rolls back partial 
assignments of other already-placed functions to improve complementary packing. 
For functions requiring distribution across multiple nodes, the algorithm performs 
fine-grained load splitting according to the function resource demand distribution. 
The algorithm outcome is a stable task-to-node mapping that balances complementary 
packing with equitable load distribution.

\subsubsection{Modeling Node Capacity and Function Demand}
\label{subsec:modeling}

A precise formulation of the placement problem requires clear representations of 
both node resource capacity and function resource demand.

\textbf{Node Resource Capacity.} We model a worker node $N_j$’s available capacity 
as a two-dimensional resource vector $\mathbf{C}_j = \left(C_j^{cpu}, C_j^{mem}\right)$, 
representing its spare CPU cores and memory (in GB). For simplicity and clarity 
of exposition, we focus on CPU and memory as the primary bottleneck resources. 
Our algorithm can be extended to incorporate other resources (\emph{e.g.}, network 
bandwidth) by increasing the dimensionality of the capacity vector. To quantify 
the efficiency of a placement, we define the \textbf{overall resource utilization} 
of a node \(N_j\) as the sum of its normalized CPU and memory utilization:
\[
U(N_j) = \frac{U_j^{\text{cpu}}}{C_{\text{total}}^{\text{cpu}}} + \frac{U_j^{\text{mem}}}{C_{\text{total}}^{\text{mem}}}
\]
where \(U_j^{\text{cpu}}\) and \(U_j^{\text{mem}}\) are the allocated resources 
on \(N_j\), and \(C_{\text{total}}^{\text{cpu}}\) and \(C_{\text{total}}^{\text{mem}}\) 
are the \textbf{total capacities of the entire cluster}. This definition inherently 
balances both resource dimensions. A placement or swapping operation is considered 
beneficial if it increases \(U(N_j)\). (The formulation can be extended with 
resource-specific weights if needed.)

\textbf{Function Resource Demand.} Modeling the resource demand of a function is 
more nuanced. We characterize it along three orthogonal dimensions:

\begin{enumerate}
    \item \textbf{Demand Ratio ($r_i$)}: The intrinsic CPU-to-memory ratio required 
    by a single execution of function $F_i$. The value of $r_i$ quantitatively 
    indicates a function’s resource affinity: a relatively small $r_i$ signifies 
    memory-intensive behavior, whereas a relatively large $r_i$ corresponds to 
    compute-intensive behavior. To reduce problem complexity, we assume this 
    ratio remains invariant across invocations of the same function.

    \item \textbf{Demand Volume ($\mathbf{V}_i$)}: The aggregate resource demand 
    vector for function $F_i$ within the scheduling window, denoted as 
    $\mathbf{V}_i = \left(V_i^{cpu}, V_i^{mem}\right)$. Here, $V_i^{cpu}$ and 
    $V_i^{mem}$ represent the predicted total CPU (\emph{e.g.}, in core-seconds) 
    and memory (\emph{e.g.}, in GB-seconds) required to process all expected 
    invocations of $F_i$ in that window, respectively. While the accurate 
    prediction of $V_i$ is orthogonal to our placement algorithm (and its accuracy 
    naturally affects overall performance), our design accepts such a predicted 
    vector as input.

    \item \textbf{Demand Distribution ($D_i$)}: The statistical distribution of 
    per-invocation resource requirements, derived from historical traces. As shown 
    in Fig.~\ref{fig:usage_patterns}, these distributions exhibit diverse shapes 
    (\emph{e.g.}, quasi-normal, long-tailed). $D_i$ is crucial for fine-grained 
    load splitting when a function must be spread across multiple nodes.
\end{enumerate}
Together, the tuple $\left<r_i, \mathbf{V}_i, D_i\right>$ forms a comprehensive 
demand profile for each function $F_i$, enabling our algorithm to reason about 
complementary placement and proportional workload distribution simultaneously.

{
% \setlength{\belowdisplayskip}{-140pt}
\begin{algorithm}[t]
    \caption{\mbox{Complementary swarming placement.}}
    \noindent \textbf{Data:}
    \begin{algorithmic}[1]
    \Statex \begin{itemize}
        \setlength{\itemindent}{.0em} % 控制缩进
        \item Pending assignment vector \(\mathbf{R} \in [0,1]^n\).
        \item Assignment matrix \(\mathbf{L} \in [0,1]^{m \times n}\).
    \end{itemize}
    \Require $ $
    \Statex \begin{itemize}
        \setlength{\itemindent}{.0em} % 控制缩进
        \item Node set \(\mathcal{N} = \{N_1, \dots, N_m\}\).
        \item Function set \(\mathcal{F} = \{F_1, \dots, F_n\}\).
    \end{itemize}
    \Ensure Optimized placement $\mathcal{P}: \mathcal{F} \times [0,1] \to \mathcal{N}$.
    
    \State \textbf{// Step 1: Initialize state \& normalize function demands}
    \State \(\mathbf{L} \gets \mathbf{0}_{m \times n}\), \(\mathbf{R} \gets \mathbf{1}_n\)
    \For{$F_i \in \mathcal{F}$}
        \State $\mathbf{V}_i \gets \mathbf{V}_i \times \frac{\sum_{j=1}^{m} \mathbf{C}_j}{\sum_{i=1}^{n} \mathbf{V}_i}$ \Comment{Scale demand proportionally}
    \EndFor
    \State
    
    \State \textbf{// Step 2: Main optimization loop}
    \While{$\exists \; F_i \text{ s.t. } R[i] > 0$} \Comment{Placement incompletement}
        \State $F_a \gets \text{RandomChoice}(\{F_i \mid R[i] > 0\})$ 
        \For{$N_j \in \mathcal{N}$}
            \State \textbf{// Assignment via free capacity}
            \State $\alpha_j^{\text{free}} \gets min(\frac{C_j^{idle\_cpu}}{V_i^{cpu}}, \frac{C_j^{idle\_mem}}{V_i^{mem}})$
            \State \textbf{// Assignment via beneficial swaps}
            \State $(\alpha_j^{\text{swap}},\; \mathcal{E}_j) \gets \text{SwapCapacity}(F_a, N_j, \mathbf{L}[j,\cdot], \mathbf{C}_j)$
            \Comment{$\mathcal{E}_j$: list of $(F_k, \beta_k)$ evicted}
            \State $\alpha_j^{\text{max}} \gets min(\alpha_j^{\text{free}} + \alpha_j^{\text{swap}}, R_a)$
            \State $\Delta U_j \gets \text{UtilGain}(F_a, \alpha_j^{\text{swap}}, \mathcal{E}_j, \mathbf{C}_j, \mathbf{C}_{\text{total}})$
        \EndFor

        \State \textbf{// Decision \& Assignment (ignoring edge cases)}
        \For{$N_j \in \mathcal{N}$, \textbf{descending} order of $\Delta U$}
            \If{$R[a] \leq 0$} \textbf{break} \EndIf
            \State $L[j, a] \gets L[j, a] + \alpha_j^{\text{max}}$, $R[a] \gets R[a] - \alpha_j^{\text{max}}$
            \For{$(F_k, \beta_k)$ in $\mathcal{E}_j$}
                \State $L[j, k] \gets L[j, k] - \beta_k$, $R[k] \gets R[k] + \beta_k$
            \EndFor
        \EndFor
    \EndWhile
    \State

    \State \textbf{// Step 3: Generate $\mathcal{P}$ from $L$ and $D_i$}
    \For{$F_i \in \mathcal{F}$}
        \State $p_{\text{start}} \gets 0.0$ \Comment To store workload percentile range
        \For{$N_j \in \mathcal{N}$, where $L[j,i] > 0$, in arbitrary order}
            \State $\mathcal{P}(F_i, [p_{\text{start}}, p_{\text{start}} + L[j,i])) \gets N_j$ 
            \State $p_{\text{start}} \gets p_{\text{start}} + L[j,i]$
        \EndFor
    \EndFor
    \State \Return $\mathcal{P}$


\end{algorithmic}
\label{alg:placement}
\end{algorithm}
}

\subsubsection{Complementary Swarming Placement Algorithm}
\label{subsec:placement-algorithm}

Algorithm~\ref{alg:placement} presents the complementary swarming placement 
algorithm. It takes as input the set of nodes $\mathcal{N}$ with their capacity 
vectors $\mathbf{C}_j$, and the set of functions $\mathcal{F}$ with their demand 
profiles $\langle r_i, \mathbf{V}_i, D_i \rangle$. The algorithm outputs a placement 
strategy $\mathcal{P}: \mathcal{F} \times [0,1] \to \mathcal{N}$ that maps function 
workloads to worker nodes. When the workload of a single function must be distributed 
across multiple worker nodes, the strategy splits the load based on the scale of 
invocation resource demand. For example, $\mathcal{P}(F_i, [0, 0.5]) = a$ and 
$\mathcal{P}(F_i, (0.5, 1.0]) = b$ means that the workload of $F_i$ is split 
evenly between nodes $a$ and $b$, where invocations with smaller resource demands 
are deployed to $a$ and those with larger demands to $b$.

\textbf{Step 1: Initialization \& Demand Normalization.}
The algorithm maintains two key data structures initialized in this step: i) a 
pending assignment vector $\mathbf{R}$, where $R[i]$ denotes the fraction of 
$F_i$ demand yet to be placed, initialized to $1$ for all functions; and ii) an 
assignment matrix $\mathbf{L}$, where $L[j,i]$ records the fraction of $F_i$ 
demand currently assigned to node $N_j$, initialized to all zeros. Furthermore, 
we scales the demand vector $\mathbf{V}_i$ of each function proportionally so 
that the aggregate demand matches the total cluster capacity 
($\sum_i \mathbf{V}_i = \sum_j \mathbf{C}_j$). This normalization ensures the 
subsequent placement operates under a balanced supply-demand premise. 

\textbf{Step 2: Iterative Optimization Loop.}
The core of the algorithm is an iterative loop that runs until all function demands 
are placed ($R[i]=0, \forall i$). In each iteration, it randomly selects a function 
$F_a$ with pending demand ($R[a] > 0$). For each node $N_j$, the algorithm evaluates 
two ways to accommodate $F_a$: i) using the idle node resources, which yields 
a capacity $\alpha_j^{\text{free}}$ computed as the minimum ratio of idle 
CPU/memory to $F_a$ demand; and ii) through beneficial swaps, where the subroutine 
$\text{SwapCapacity}$ attempts to evict already-placed functions in order to host 
as much of $F_a$ demand as possible, provided that doing so improves the node 
utilization $U_j$ (detailed in Algorithm~2). $\text{SwapCapacity}$ returns both 
the maximum $F_a$ demand fraction $\alpha_j^{\text{swap}}$ that can be hosted by 
function swaping, and a list $\mathcal{E}_j$ of evicted functions, each with the 
evicted load fraction $\beta$. The total accommodation capacity for $F_a$ on 
$N_j$ is then $\alpha_j^{\text{max}} = \min(\alpha_j^{\text{free}}+\alpha_j^{\text{swap}}, R[a])$.

After evaluating all nodes, the algorithm sorts them in descending order of the 
utilization gain $\Delta U_j$ and assigns $F_a$ pending demand greedily. For each 
node in this order, it assigns up to $\alpha_j^{\text{max}}$ of $F_a$ demand, 
updates $\mathbf{L}$ and $\mathbf{R}$, and simultaneously processes the evictions 
in $\mathcal{E}_j$: for each function $F_k$ in $\mathcal{E}_j$, the algorithm 
reducing the evicted fraction $\beta_k$ in $\mathbf{L}[j, k]$ and adds $\beta_k$ 
back to $F_k$ pending demand $\mathbf{R}[k]$. This process continues until $F_a$ 
demand is fully placed ($R[a]=0$). Because the total function demand was scaled 
to match the total cluster capacity in Step 1, it is guaranteed that $F_a$ demand 
can eventually be fully accommodated, and the loop will not terminate with 
$R[a] > 0$. To ensure the loop convergence speed, the algorithm skips nodes whose 
utilization gain $\Delta U_j$ is below a predefined empirical threshold 
(\emph{e.g.}, 1\%).

\textbf{Step 3: Fine-Grained Placement Generation.}
After the assignment matrix $\mathbf{L}$ is determined, the algorithm converts the proportional assignments into a precise placement mapping $\mathcal{P}$. For each function $F_i$, it sequentially allocates contiguous percentile intervals from $[0, 1]$ to each node $N_j$ that hosts $F_i$, with the length of each interval equal to the assigned proportion $L[j,i]$. It directly yields the mapping $\mathcal{P}$: a request of function $F_i$ whose resource demand falls at percentile $p$ within $F_i$ historical distribution $D_i$ is then dispatched to the node $N_j$ whose allocated interval covers $p$.

% 3. 任务分配方案 (Task Assignment)
%   - 算法输出的“函数-节点”映射如何转化为实时调度决策。

% 4. 启发式算法流程 (Heuristic Algorithm)
%   - 说明SwapCapacity方法
%   - **收敛性**:解释为何在可控步长下,算法能在有限迭代内达到稳定状态。

% 5. 算法属性保证 (Algorithmic Guarantees)
%   - **低开销**:“部分更新”机制如何运作(例如,仅重新计算受影响节点的分配,或定期而非每请求触发)。

\subsection{Monotonic Scale-up Task Scheduling}
\label{sec:design:scheduling}

% 1. 调度器目标与总览 (Scheduler Objective & Overview)
%   - 重申目标:解决函数内需求波动,实现无重启细粒度资源调整。
%   - 总览架构图:展示双队列、多沙箱、与K8s运行时交互的组件图。

% 2. 调度器核心结构 (Core Scheduler Structure)
%   - **双任务队列**:明确两个队列的作用(例如,Q1: 等待分配沙箱的任务;Q2: 已分配待执行的任务)。
%   - **排序规则**:队列按预测资源需求单调递增排序。
%   - **多沙箱均分优先级结构**:如何将不同函数的请求分配到不同沙箱,以及如何在沙箱间平衡负载。

% 3. 优先级设计 (Priority Design)
%   - 定义优先级函数,综合考虑等待时间、冷启动开销、资源需求。
%   - 解释如何通过“在优先级约束下策略性重排序”来延长队列,从而提升利用率和减少冷启动。

% 4. 事件驱动调度逻辑 (Event-driven Scheduling Logic)
%   - 列出关键事件:新请求到达、沙箱资源调整完成、沙箱空闲、资源回收触发。
%   - 描述调度器对每个事件的反应逻辑(可配流程图或状态机)。